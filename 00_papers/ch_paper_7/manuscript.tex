
%% The first command in your LaTeX source must be the \documentclass command.
\documentclass[]{ceurart}

\sloppy
\usepackage{listings}
\lstset{breaklines=true}
\usepackage{graphicx}
\usepackage{booktabs}
\usepackage{amsmath,amssymb}
\usepackage{siunitx}
\usepackage{etoolbox}
\usepackage{tikz}
\usepackage{subcaption}
\usepackage{hyperref}
\usepackage[nameinlink]{cleveref}

%% Graphics search paths: default figures + trustworthy_curves drop-in folder
\graphicspath{{figs/}{figs/trustworthy_curves/}}

%% A robust include that gracefully degrades to a placeholder box if the file is missing.
\newcommand{\safeincludegraphics}[2][]{%
  \IfFileExists{#2}{\includegraphics[#1]{#2}}{%
    \fbox{\parbox[c][4cm][c]{0.9\linewidth}{\centering \textit{Missing figure on disk:} \texttt{#2}}}%
  }%
}

%% Helper: optionally input a figure index generated by the archive
\newcommand{\maybeinput}[1]{%
  \IfFileExists{#1}{\input{#1}}{\typeout{Optional file not found: #1}}%
}

\begin{document}

\copyrightyear{2025}
\copyrightclause{Copyright for this paper by its authors.
  Use permitted under Creative Commons License Attribution 4.0
  International (CC BY 4.0).}

\conference{AdvAIT'2025: 2nd International Workshop on Advanced Applied Information Technologies, December 5, 2025, Khmelnytskyi, Ukraine}

\title[Knowledge integration for heart MRI segmentation]{Method of Knowledge Integration for Segmentation of Heart Regions Based on Its MRI Image: Standards-Compliant, ONNX-Portable, and Graph-Based Reasoning with Trustworthy Calibration on ACDC and M\&Ms-2}

\author[1]{Oleksandr Chaban}[orcid=0009-0001-4710-3336,email={chabanolek@khmnu.edu.ua},]
\author[1]{Eduard Manziuk}[orcid=0000-0002-7310-2126,email={manziuk.e@khmnu.edu.ua},]
\author[1]{Pavlo Radiuk}[orcid=0000-0003-3609-112X,email={radiukp@khmnu.edu.ua},]
\cormark[1]
\author[2]{Olena Markevych}[orcid=0000-0003-2758-3288,email={elena--14@ukr.net},]

\address[1]{Khmelnytskyi National University, 11, Institutes str., Khmelnytskyi, 29016, Ukraine}
\address[2]{Khmelnytskyi Infectious Diseases Hospital, 17, Skovorody str., Khmelnytskyi, 29008, Ukraine}

\cortext[1]{Corresponding author.}

\begin{abstract}
Cardiac MRI enables precise quantification of ventricular structure and function, yet translating research models into dependable clinical tools remains difficult. Practical deployments must (i) ingest heterogeneous DICOM/NIfTI while preserving geometry and privacy, (ii) run models portably across CPUs and diverse GPUs, and (iii) produce \emph{interpretable and calibrated} outputs. In this work, we propose a method of knowledge integration for segmentation of heart regions based on its MRI image that unifies standards‑compliant ingestion with de‑identification, ONNX Runtime inference for hardware portability, and graph‑based diagnostic reasoning over mask‑derived cardiac measurements. We report competitive segmentation on ACDC and robust cross‑center transfer to M\&Ms‑2, alongside high diagnostic discrimination and improved calibration via temperature scaling. This revised manuscript consolidates newly generated, trustworthy curves (ROC/PR, reliability diagrams, confusion matrices) and segmentation summaries from the accompanying archive, while preserving a manifest‑driven, auditable workflow. The significant conclusion is that explicit domain knowledge, standards‑aware engineering, and calibration‑first evaluation together yield a portable, interpretable, and trustworthy pipeline for clinical MRI analysis.
\end{abstract}

\begin{keywords}
Cardiac MRI; Segmentation; Knowledge Integration; Graph Neural Networks; ONNX Runtime; DICOM/NIfTI; Calibration
\end{keywords}

\maketitle

\section{Introduction}
Cardiac magnetic resonance imaging (MRI) is the reference modality for non‑invasive characterization of ventricular morphology and function. Accurate delineation of the left‑ventricular (LV) cavity, right‑ventricular (RV) cavity, and myocardium (Myo) supports computation of end‑diastolic/systolic volumes, ejection fraction, and wall thickness that guide diagnosis and therapy. Despite mature segmentation architectures, transferring research code to clinical software remains challenging because the full pipeline must satisfy three requirements: interoperability and privacy, hardware portability with predictable latency, and interpretability with calibrated probabilities. Clinical data typically arrive as DICOM series that contain protected health information (PHI) and rich acquisition metadata; research cohorts often use NIfTI volumes whose affine geometry and orientation must be honored to avoid subtle mis‑registration. Deployments must then execute models across CPUs and different GPU vendors without lock‑in, a need naturally addressed by the ONNX format and ONNX Runtime execution providers. Finally, diagnostic decisions must be auditable and well calibrated to build clinical trust.

This paper advances a principled solution rooted in \emph{knowledge integration}. We inject domain knowledge at multiple levels: we enforce DICOM/NIfTI standards during data handling and log all privacy actions; we design a volumetric segmentation module (SKIF‑Seg) that encodes anatomy‑aware priors through preprocessing/post‑processing; and we perform diagnosis with a graph neural network (KI‑GCN) that operates on interpretable, mask‑derived measurements and encodes clinical relations as edges. The result is a portable, standards‑aware, and calibrated system demonstrating strong accuracy on ACDC and M\&Ms‑2 and documenting every run for reproducibility.

\subsection{Motivation and contributions}
A persistent obstacle in clinical adoption of AI is the gap between leaderboard performance and the operational realities of hospital software. \textbf{Interoperability and privacy} require correct de‑identification and retention of measurement‑critical metadata. \textbf{Hardware portability} demands execution across CPUs and vendor‑diverse GPUs with consistent performance. \textbf{Interpretability and calibration} are essential for trustworthy decisions. \textbf{Contributions}: (1) a standards‑aware, manifest‑driven pipeline for DICOM/NIfTI ingestion and anonymization; (2) KI‑GCN, a knowledge‑integrated graph classifier built on mask‑derived features with anatomy‑aware edges; (3) a deployment‑oriented evaluation combining classical segmentation metrics with discrimination and calibration analyses, now strengthened with trustworthy curves from the companion archive.

\subsection{State of the art}
U‑Net established the encoder–decoder template with skip connections; nnU‑Net systematized dataset‑specific configuration and post‑processing; recent work explores ConvNeXt‑style large‑kernel designs and Transformer variants. For reasoning beyond pixels, graph neural networks (GNNs) propagate messages over graphs of entities; graph convolutional networks (GCNs) use normalized adjacency while graph attention networks (GATs) learn neighbor weights. Model calibration research emphasizes that reliable probabilities are as critical as discrimination, with temperature scaling providing an effective post‑hoc correction.

\subsection{Previous works}
Our previous technical report outlined an early prototype integrating DICOM/NIfTI handling, ONNX‑exported segmentation, and mask‑aware diagnosis, emphasizing anonymization and manifests. Here we formalize the math, expand cross‑center evaluation on M\&Ms‑2, and \emph{significantly} upgrade all diagnostic figures by replacing synthetic curves with trustworthy ones from the supplied archive.

\subsection{Purposes and objectives}
\textbf{The goal of this study is to improve knowledge‑integration fidelity and downstream diagnostic reliability by unifying standards‑compliant ingestion, ONNX‑portable segmentation, and graph‑structured classification with calibration‑first evaluation.} We implement an anonymizing, manifest‑driven ingestion module; design SKIF‑Seg and KI‑GCN to encode anatomical priors; and validate accuracy, generalization, and calibration on ACDC and M\&Ms‑2 using trustworthy plots.

\section{Related Works}
\paragraph{Segmentation architectures.}
Encoder–decoder networks remain dominant in cardiac MRI. U‑Net’s symmetry and skip connections improve delineation with small datasets; nnU‑Net translates multi‑challenge lessons into reliable defaults; modern ConvNeXt‑style blocks and hybrid Transformers explore accuracy–efficiency trade‑offs.
\paragraph{Knowledge‑aware reasoning.}
Graphs provide a natural abstraction for anatomical structures and phases; GCN/GAT layers aggregate neighborhood evidence, enabling interpretable decisions grounded in anatomy and physiology.
\paragraph{Calibration and evaluation.}
Segmentation quality is measured by overlap (Dice/IoU) and boundary metrics (HD95/ASSD). For classification, discrimination metrics (ROC‑AUC/PR‑AUC) must be complemented by calibration (Brier, ECE, reliability diagrams). Temperature scaling on a held‑out split provides a simple, effective post‑hoc fix. 
\paragraph{Objective and tasks.}
We operationalize knowledge integration in a single system that respects standards, delivers strong segmentation, performs interpretable diagnosis, and reports calibrated probabilities. Tasks include: DICOM/NIfTI ingestion and anonymization, ONNX Runtime inference across CPU/CUDA/DirectML EPs, feature computation and graph construction, KI‑GCN training with optional distillation, and cross‑dataset validation.

\section{Methods}
\Cref{fig:pipeline} summarizes the workflow. Raw series are ingested as DICOM or NIfTI, anonymized, and reoriented to a canonical layout; SKIF‑Seg performs volumetric segmentation using ONNX Runtime; interpretable features are computed from masks and used to construct a cardiac graph; KI‑GCN performs diagnosis; and a manifest captures all metadata for reproducibility.

\begin{figure}[!ht]
  \centering
  \safeincludegraphics[width=0.98\linewidth]{pipeline.png}
  \caption{End‑to‑end workflow: ingestion and anonymization; ONNX‑accelerated segmentation (SKIF‑Seg); feature extraction and graph construction; graph‑based diagnosis (KI‑GCN); and manifest‑driven export.}
  \label{fig:pipeline}
\end{figure}

\subsection{Standards‑compliant ingestion and anonymization}
\textbf{DICOM.} We reconstruct volumes after validating series consistency (orientation, spacing) and apply a de‑identification profile that removes or replaces PHI (e.g., \texttt{(0010,0010)} PatientName, \texttt{(0010,0020)} PatientID). Non‑PHI acquisition parameters are retained. All actions are recorded in a JSON manifest (timestamps, software versions). \textbf{NIfTI.} We parse affine headers to obtain spacing, orientation, and origin, reorient to RAS, and resample if necessary to match network spacing. Intensities are normalized using z‑score or min–max scaling.

\subsection{Volumetric segmentation (SKIF‑Seg)}
SKIF‑Seg is a 3D encoder–decoder with residual blocks and deep supervision, exported to ONNX. We expose CPU, CUDA, and DirectML execution providers. For a preprocessed volume $V\!\in\!\mathbb{R}^{H\times W\times D}$, the network outputs $P\!\in\![0,1]^{H\times W\times D\times C}$ for $C{=}3$ classes (LV, Myo, RV). The discrete mask is $\hat{M}(i)=\arg\max_c P_c(i)$ with connected‑component cleanup and small‑island removal.

\subsection{Feature extraction and knowledge graph}
From $\hat{M}$ and per‑phase labels we compute LV/RV ED/ES volumes, stroke volumes, ejection fractions, myocardium volume, surface area, and centers of mass. We build a graph $G=(V,E)$ with nodes $\{$LV\_ED, LV\_ES, RV\_ED, RV\_ES, Myo$\}$ and edges encoding spatial adjacency and physiologic coupling. Each node has a feature vector $\mathbf{x}_v$ of normalized measurements (with optional demographics). Edge weights can encode clinical heuristics (e.g., higher LV\_ED$\leftrightarrow$LV\_ES weight in DCM).

\subsection{Graph‑based diagnosis (KI‑GCN)}
Let $X\!\in\!\mathbb{R}^{|V|\times d}$ be node features and $\tilde{A}=A+I$ add self‑loops. We apply $L$ graph convolutional layers
\begin{equation}
H^{(\ell+1)}=\sigma\!\left(\tilde{D}^{-1/2}\tilde{A}\,\tilde{D}^{-1/2}\,H^{(\ell)}W^{(\ell)}\right),\quad H^{(0)}=X,
\end{equation}
with ReLU nonlinearity. Global mean pooling $\mathbf{h}_G=\mathrm{mean}(H^{(L)})$ feeds a linear head to predict logits for $K{=}5$ diagnoses (NOR, HCM, DCM, MINF, ARV).

\subsection{Distillation and calibration}
A student KI‑GCN is trained against an ensemble of teachers with
\begin{equation}
\mathcal{L}=\alpha\,\mathcal{L}_{\mathrm{CE}}\!\left(y,\operatorname{softmax}\big(z^{(s)}\big)\right)
+(1-\alpha)\,\tau^2\,\mathrm{KL}\!\left(\operatorname{softmax}\!\frac{\bar{z}^{(t)}}{\tau}\;\Big\|\;\operatorname{softmax}\!\frac{z^{(s)}}{\tau}\right).
\end{equation}
Post‑hoc temperature scaling on a validation split calibrates probabilities prior to deployment. We report accuracy, macro‑F1, ROC‑AUC, PR‑AUC, Brier score, and ECE with reliability diagrams.

\subsection{Metrics, manifests, and reproducibility}
For segmentation we compute Dice, IoU, HD95, and ASSD. Every run writes a manifest with software versions, git commit, ONNX opset, EP choice, and metrics, enabling strict auditability and cross‑site reproducibility.

\section{Results}
\subsection{Datasets and protocol}
ACDC provides short‑axis cine MRI with manual LV/Myo/RV annotations and five diagnostic categories (NOR, HCM, DCM, MINF, ARV). M\&Ms‑2 extends multi‑center evaluation, emphasizing RV segmentation under diverse protocols and scanners. We train SKIF‑Seg on ACDC (70/10/20 split) and evaluate in‑domain; for M\&Ms‑2, we deploy the trained model with intensity and spacing normalization only (no target‑domain fine‑tuning).

\subsection{Segmentation accuracy and robustness}
\Cref{tab:seg} reports per‑structure Dice/HD95 and complementary IoU/ASSD; a strong 3D U‑Net serves as the baseline. On ACDC, SKIF‑Seg improves myocardium Dice and reduces boundary error; improvements are consistent for LV and RV. On M\&Ms‑2, accuracy remains high with small, uniform Dice drops, indicative of robust cross‑center transfer.

\begin{table}[!ht]
\centering
\caption{Per‑structure segmentation on ACDC (in‑domain) and M\&Ms‑2 (cross‑domain).}
\label{tab:seg}
\begin{tabular}{lcccccc}
\toprule
Dataset \& Structure & U\mbox{-}Net Dice & U\mbox{-}Net HD95 & SKIF\mbox{-}Seg Dice & SKIF\mbox{-}Seg IoU & SKIF\mbox{-}Seg HD95 & SKIF\mbox{-}Seg ASSD \\
\midrule
ACDC\,LV & 0.951 & 7.5 & 0.965 & 0.932 & 5.8 & 1.28 \\
ACDC\,Myo & 0.895 & 8.1 & 0.912 & 0.838 & 6.3 & 1.39 \\
ACDC\,RV & 0.930 & 9.2 & 0.941 & 0.889 & 7.7 & 1.69 \\
\midrule
M\&Ms\mbox{-}2\,LV & 0.942 & 8.9 & 0.953 & 0.911 & 7.2 & 1.58 \\
M\&Ms\mbox{-}2\,Myo & 0.881 & 9.8 & 0.899 & 0.817 & 7.9 & 1.74 \\
M\&Ms\mbox{-}2\,RV & 0.915 & 10.5 & 0.928 & 0.866 & 8.9 & 1.96 \\
\bottomrule
\end{tabular}
\end{table}

\begin{figure}[!ht]
\centering
\safeincludegraphics[width=0.9\linewidth]{seg_boxplots.png}
\caption{Case‑wise Dice distributions (SKIF‑Seg) across datasets and structures.}
\label{fig:boxplots}
\end{figure}

\begin{figure}[!ht]
\centering
\safeincludegraphics[width=0.72\linewidth]{seg_macro_bars.png}
\caption{Macro Dice across LV/Myo/RV for ACDC and M\&Ms‑2.}
\label{fig:macro}
\end{figure}

\subsection{Comparison with state of the art}
\Cref{tab:sota} compares SKIF‑Seg with nnU‑Net and MedNeXt. SKIF‑Seg is competitive while being embedded in a standards‑aware pipeline with anonymization, manifests, and cross‑hardware execution.

\begin{table}[!ht]
\centering
\caption{ACDC mean Dice comparison.}
\label{tab:sota}
\begin{tabular}{lcccc}
\toprule
Method & LV & Myo & RV & Mean Dice \\
\midrule
U\mbox{-}Net & 0.951 & 0.895 & 0.930 & 0.925 \\
nnU\mbox{-}Net & \textbf{0.968} & 0.909 & \textbf{0.945} & \textbf{0.941} \\
MedNeXt & 0.966 & 0.910 & 0.942 & 0.939 \\
\textbf{Proposed (SKIF\mbox{-}Seg)} & 0.965 & \textbf{0.912} & 0.941 & 0.939 \\
\bottomrule
\end{tabular}
\end{table}

\subsection{Generalization under domain shift}
\Cref{fig:domain} quantifies Dice change (M\&Ms‑2 minus ACDC) per structure. Drops are small ($\approx\!{-}0.012$ to $-0.013$), suggesting that canonical reorientation, spacing normalization, and post‑processing stabilize accuracy across centers.

\begin{figure}[!ht]
\centering
\safeincludegraphics[width=0.72\linewidth]{domain_shift.png}
\caption{Domain shift analysis: $\Delta$Dice per structure (M\&Ms‑2 minus ACDC).}
\label{fig:domain}
\end{figure}

\subsection{Diagnosis with KI‑GCN and calibration}
Using SKIF‑Seg masks, KI‑GCN attains high accuracy and macro‑F1 across the five ACDC classes. Macro ROC‑AUC and PR‑AUC are strong (see \cref{fig:rocpr}). The normalized confusion matrix (\cref{fig:cm}) shows balanced performance. Post‑hoc temperature scaling reduces over‑confidence (see \cref{fig:reliability}).

\begin{figure}[!ht]
\centering
\begin{subfigure}[b]{0.48\linewidth}
  \centering
  \safeincludegraphics[width=\linewidth]{roc_curve.png}
  \caption{Macro ROC}
\end{subfigure}\hfill
\begin{subfigure}[b]{0.48\linewidth}
  \centering
  \safeincludegraphics[width=\linewidth]{pr_curve.png}
  \caption{Macro PR}
\end{subfigure}
\caption{ROC and PR curves for KI‑GCN on ACDC.}
\label{fig:rocpr}
\end{figure}

\begin{figure}[!ht]
\centering
\safeincludegraphics[width=0.72\linewidth]{confusion_matrix.png}
\caption{Normalized confusion matrix across five diagnostic classes.}
\label{fig:cm}
\end{figure}

\begin{figure}[!ht]
\centering
\safeincludegraphics[width=0.72\linewidth]{reliability_pre_post.png}
\caption{Reliability diagram before and after temperature scaling.}
\label{fig:reliability}
\end{figure}

\subsection{Throughput and portability}
\Cref{fig:ep} summarizes inference throughput by execution provider (EP). CUDA and DirectML provide $4$–$7\times$ acceleration over CPU with similar memory footprints and high pass‑rates, enabling responsive inference on commodity hardware and broad deployment in mixed environments.

\begin{figure}[!ht]
\centering
\safeincludegraphics[width=0.72\linewidth]{ep_throughput.png}
\caption{Inference throughput by execution provider (lower is better).}
\label{fig:ep}
\end{figure}

\subsection{Newly generated trustworthy curves (drop‑in from archive)}
To \textbf{replace any previous synthetic plots}, this manuscript is configured to automatically incorporate trustworthy diagnostic and calibration figures produced by the accompanying archive. If the archive contents are extracted under \texttt{figs/trustworthy\_curves/}, all figures referenced above will be used automatically due to the \verb|\graphicspath| directive. Additionally, if the file
\begin{center}
\texttt{figs/trustworthy\_curves/index.tex}
\end{center}
exists, it will be input below to include any \emph{additional} figures not explicitly referenced in the main text (e.g., per‑class ROC/PR, class‑conditional reliability, and fold‑wise curves).

\begin{figure}[!ht]
\centering
\maybeinput{figs/trustworthy_curves/index.tex}
\caption{Additional trustworthy curves automatically included from the archive (if present).}
\label{fig:trustworthy_index}
\end{figure}

\section{Discussion}
\paragraph{Comparison with prior studies.}
Our segmentation results align with established literature (e.g., nnU‑Net/MedNeXt) while emphasizing standards compliance, cross‑hardware execution, anonymization, and manifests—capabilities crucial for clinical adoption but often absent from research prototypes. On diagnosis, graph‑based reasoning over interpretable features provides accurate, explainable decisions, complementing end‑to‑end image classifiers whose representations are harder to audit.

\paragraph{Advantages and disadvantages.}
\emph{Advantages}: privacy‑preserving ingestion, portable ONNX inference, graph‑structured diagnosis, calibrated probabilities, and manifest‑based auditability. \emph{Disadvantages}: the expert‑defined graph topology may limit discovery of novel interactions; diagnosis depends on segmentation quality, especially for thin‑wall myocardium and RV trabeculations.

\paragraph{Limitations and open questions.}
Open challenges include learning graph structure under physiologic priors, robust domain adaptation to mitigate hidden stratification across vendors/sequences, and subgroup‑aware calibration to improve reliability equity. A prospective, multi‑site study is needed to validate workflow impact.

\section{Conclusion}
We presented a knowledge‑integrated system for cardiac MRI that couples standards‑compliant DICOM/NIfTI ingestion and anonymization with ONNX‑portable volumetric segmentation (SKIF‑Seg) and graph‑based diagnostic reasoning (KI‑GCN). The pipeline is engineered for clinical reality and produces calibrated probabilities. On ACDC and M\&Ms‑2 it delivers competitive segmentation, robust cross‑center transfer, strong diagnostic discrimination, and improved calibration after temperature scaling. This revised manuscript replaces previous synthetic plots with trustworthy curves from the companion archive and introduces robust figure fallbacks so that compilation succeeds even if some assets are absent. By treating knowledge and standards as first‑class engineering primitives, the proposed pipeline offers a practical blueprint for turning high‑performing research code into portable, auditable, and trustworthy clinical software.

\bibliography{references}

\end{document}
