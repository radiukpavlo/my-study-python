\subsection{Solving Systems using $LU$ Factorization}

One reason people care about the $LU$ factorization
\index{LU factorization!solving systems} is it allows the quick solution of
systems of equations. Here is an example.

\begin{example}{$LU$ factorization to Solve Equations}{}
Suppose you want to find the solutions to
\begin{equation*}
\leftB
\begin{array}{rrrr}
1 & 2 & 3 & 2 \\
4 & 3 & 1 & 1 \\
1 & 2 & 3 & 0
\end{array}
\rightB \leftB
\begin{array}{c}
x \\
y \\
z \\
w
\end{array}
\rightB =\leftB
\begin{array}{c}
1 \\
2 \\
3
\end{array}
\rightB .
\end{equation*}
\end{example}

\begin{solution}

Of course one way is to write the augmented matrix and grind away. However,
this involves more row operations than the computation of the $LU$
factorization and it turns out that the $LU$ factorization can give the
solution quickly. Here is how. The following is an $LU$ factorization for
the matrix. 
\begin{equation*}
\leftB 
\begin{array}{rrrr}
1 & 2 & 3 & 2 \\ 
4 & 3 & 1 & 1 \\ 
1 & 2 & 3 & 0
\end{array}
\rightB \allowbreak =\allowbreak \leftB 
\begin{array}{rrr}
1 & 0 & 0 \\ 
4 & 1 & 0 \\ 
1 & 0 & 1
\end{array}
\rightB \leftB 
\begin{array}{rrrr}
1 & 2 & 3 & 2 \\ 
0 & -5 & -11 & -7 \\ 
0 & 0 & 0 & -2
\end{array}
\rightB .
\end{equation*}

Let $UX=Y$ and consider $LY=B$ where in this case, $B=\leftB 1,2,3\rightB ^{T}$. Thus 
\begin{equation*}
\allowbreak \leftB 
\begin{array}{rrr}
1 & 0 & 0 \\ 
4 & 1 & 0 \\ 
1 & 0 & 1
\end{array}
\rightB \leftB 
\begin{array}{c}
y_{1} \\ 
y_{2} \\ 
y_{3}
\end{array}
\rightB =\leftB 
\begin{array}{c}
1 \\ 
2 \\ 
3
\end{array}
\rightB
\end{equation*}
which yields very quickly that $Y=\leftB 
\begin{array}{r}
1 \\ 
-2 \\ 
2
\end{array}
\rightB \allowbreak .$ 

Now you can find $X$ by solving $UX=Y$. Thus in this case, 
\begin{equation*}
\leftB 
\begin{array}{rrrr}
1 & 2 & 3 & 2 \\ 
0 & -5 & -11 & -7 \\ 
0 & 0 & 0 & -2
\end{array}
\rightB \leftB 
\begin{array}{c}
x \\ 
y \\ 
z \\ 
w
\end{array}
\rightB =\leftB 
\begin{array}{r}
1 \\ 
-2 \\ 
2
\end{array}
\rightB
\end{equation*}
which yields 
\begin{equation*}
X=\leftB 
\begin{array}{c}
-
\vspace{0.05in}\frac{3}{5}+\vspace{0.05in}\frac{7}{5}t \\ 
\vspace{0.05in}\frac{9}{5}-\vspace{0.05in}\frac{11}{5}t \\ 
t \\ 
-1
\end{array}
\rightB ,\enspace t\in \mathbb{R}\text{.}
\end{equation*}

\end{solution}