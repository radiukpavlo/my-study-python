\section{Roots of Complex Numbers}

\begin{outcome}
\begin{enumerate}
\item[A.] Understand De Moivre's theorem and be able to use it to find the
roots of a complex number.
\end{enumerate}
\end{outcome}

A fundamental identity is the
\index{De Moivre's theorem} formula of De Moivre with which we begin this section.

\begin{theorem}{De Moivre's Theorem}{demoivretheorem}
For any positive integer $n$, we have
\begin{equation*}
\left( e^{i \theta} \right)^n = e^{i n \theta}
\end{equation*}
Thus for any real number $r>0$ and any positive integer $n$, we have:
\begin{equation*}
\left( r\left( \cos \theta+i\sin \theta\right) \right) ^{n}=r^{n}\left( \cos n \theta +i\sin
n\theta\right) 
\end{equation*}
\end{theorem}

\begin{proof}
 The proof is by induction on $n$. It is clear the formula holds if $n=1.$ Suppose it is true
for $n.$ Then, consider $n+1$.
\begin{equation*}
\left( r\left( \cos \theta+i\sin \theta\right) \right) ^{n+1}=\left( r\left( \cos
\theta+i\sin \theta\right) \right) ^{n}\left( r\left( \cos \theta+i\sin \theta\right) \right)
\end{equation*}
which by induction equals
\begin{eqnarray*}
&=&r^{n+1}\left( \cos n\theta+i\sin n\theta\right) \left( \cos \theta+i\sin \theta\right) \\
&=& r^{n+1}\left( \left( \cos n\theta\cos \theta-\sin n\theta\sin \theta\right) +i\left( \sin
n\theta\cos \theta+\cos n\theta\sin \theta\right) \right)\\
&=&r^{n+1}\left( \cos \left( n+1\right) \theta+i\sin \left( n+1\right) \theta\right)
\end{eqnarray*}
by the formulas for the cosine and sine of the sum of two angles. 
\end{proof}

The process used in the previous proof, called {\em mathematical
induction\em} is very powerful in Mathematics and Computer Science
and explored in more detail in the Appendix.

Now, consider a corollary of Theorem \ref{thm:demoivretheorem}.

\begin{corollary}{Roots of Complex Numbers}{rootscomplexnumbers}
Let $z$ be a non zero complex number.
\index{complex numbers!roots}Then there are always exactly $k$ many  $k^{th}$
roots of $z$ in $\mathbb{C}$.
\end{corollary}

\begin{proof}
Let $z=a+bi$ and let $z=\left\vert z\right\vert \left( \cos
\theta+i\sin \theta\right) $ be the polar form of the complex number. By De Moivre's
theorem, a complex number
\begin{equation*}
w= r e^{i \alpha} = r\left( \cos \alpha +i\sin \alpha \right) 
\end{equation*}
is a $k^{th}$ root of $z$ if and only if
\begin{equation*}
w^k = (r e^{i \alpha})^k = r^k e^{ik\alpha} = r^{k}\left( \cos k\alpha +i\sin k\alpha \right) =\left\vert z\right\vert
\left( \cos \theta+i\sin \theta\right) 
\end{equation*}
This requires $r^{k}=\left\vert z\right\vert $ and so $r=\left\vert
z\right\vert ^{1/k}$. Also, both $\cos \left( k\alpha \right) =\cos \theta$ and
$\sin \left( k\alpha \right) =\sin \theta.$ This can only happen if
\begin{equation*}
k\alpha =\theta+2 \ell \pi
\end{equation*}
for $\ell$ an integer. Thus
\begin{equation*}
\alpha =
\frac{\theta+2 \ell \pi }{k},\; \ell = 0, 1, 2, \cdots, k-1 
\end{equation*}
and so the $k^{th}$ roots of $z$ are of the form
\begin{equation*}
\left\vert z\right\vert ^{1/k}\left( \cos \left( \frac{\theta+2 \ell \pi }{k}\right)
+i\sin \left( \frac{\theta+2 \ell \pi }{k}\right) \right) ,\;\ell = 0, 1, 2, \cdots, k-1 
\end{equation*}
Since the cosine and sine are periodic of period $2\pi ,$ there are exactly $
k$ distinct numbers which result from this formula. 
\end{proof}

The procedure for finding the $k$ $k^{th}$ roots of $z \in \mathbb{C}$ is as follows.

\begin{procedure}{Finding Roots of a Complex Number}{findingkthroots}
Let $w$ be a complex number. We wish to find the $n^{th}$ roots of $w$, that is all $z$ such that $z^n = w$. 

There are $n$ distinct $n^{th}$ roots and they can be found as follows:. 
 
\begin{enumerate}
\item Express both $z$ and $w$ in polar form $z=re^{i\theta}, w=se^{i\phi}$. Then $z^n = w$ becomes:
\[
(re^{i\theta})^n = r^n e^{i n \theta} = se^{i\phi}
\]
We need to solve for $r$ and $\theta$. 
\item Solve the following two equations:
\begin{eqnarray*}
r^n &=& s 
\end{eqnarray*}
\begin{eqnarray}
e^{i n \theta} &=& e^{i \phi}
\label{rootseqns}
\end{eqnarray}
\item The solutions to $r^n = s$ are given by $r = \sqrt[n]{s}$. 

\item The solutions to $e^{i n \theta} = e^{i \phi}$ are given by:
\[
n\theta = \phi + 2\pi \ell,  \; \mbox{for} \; \ell = 0,1,2, \cdots, n-1
\]
or
\[
\theta = \frac{\phi}{n} + \frac{2}{n} \pi \ell, \; \mbox{for} \; \ell = 0,1,2, \cdots, n-1 
\]
\item
Using the solutions $r, \theta$ to the equations given in (\ref{rootseqns})
construct the $n^{th}$ roots of the form $z = re^{i\theta}$.  
\end{enumerate}
\end{procedure}

Notice that once the roots are obtained in the final step, they can then be converted to standard form if necessary. Let's consider an example of this concept. Note that according to Corollary \ref{cor:rootscomplexnumbers}, 
there are exactly $3$ cube roots of a complex number.

\begin{example}{Finding Cube Roots}{cuberoots}
Find the three cube roots of $i.$ In other words find all $z$ such that $z^3 = i$. 
\end{example}

\begin{solution}
First, convert each number to polar form: $z = re^{i\theta}$ and $i = 1 e^{i \pi/2}$. The equation now becomes
\[
(re^{i\theta})^3 = r^3 e^{3i\theta} = 1 e^{i \pi/2}
\]
Therefore, the two equations that we need to solve are $r^3 = 1$ and $3i\theta = i \pi/2$. Given that $r \in \mathbb{R}$ and $r^3 = 1$ it follows that $r=1$. 

Solving the second equation is as follows. First divide by $i$. Then, since the argument of $i$ is not unique we write $3\theta = \pi/2 + 2\pi\ell$ for $\ell = 0,1,2$. 
\begin{eqnarray*}
3\theta &=& \pi/2 + 2\pi\ell \; \mbox{for} \; \ell = 0,1,2 \\
\theta &=& \pi/6 + \frac{2}{3} \pi\ell \; \mbox{for} \; \ell = 0,1,2 
\end{eqnarray*}

For $\ell = 0$:
\[
\theta = \pi/6 + \frac{2}{3} \pi (0) = \pi/6
\]

For $\ell = 1$:
\[
\theta = \pi/6 + \frac{2}{3} \pi(1) = \frac{5}{6} \pi
\]

For $\ell = 2$:
\[
\theta = \pi/6 + \frac{2}{3} \pi(2) = \frac{3}{2} \pi
\]

Therefore, the three roots are given by \[
1e^{i \pi/6}, 1e^{i \frac{5}{6}\pi}, 1e^{i \frac{3}{2}\pi}
\]

Written in standard form, these roots are, respectively,
\[
\frac{\sqrt{3}}{2} + i \frac{1}{2}, -\frac{\sqrt{3}}{2} + i \frac{1}{2}, -i
\]

\end{solution}

The ability to find $k^{th}$ roots can also be used to factor some
polynomials.

\begin{example}{Solving a Polynomial Equation}{solvingpolynomial}
Factor the polynomial $x^{3}-27.$
\end{example}

\begin{solution}
First find the cube roots of 27. By the above procedure
\index{polynomials!factoring}, these cube roots
are $3,3\left( \displaystyle
\frac{-1}{2}+i\displaystyle\frac{\sqrt{3}}{2}\right) ,$ and $3\left( 
\displaystyle\frac{-1}{2}-i\displaystyle\frac{\sqrt{3}}{2}\right) $. You may wish to verify 
this using the above steps.

Therefore, $x^{3}-27 =$
\begin{equation*}
 \left( x-3\right) \left( x-3\left( \frac{-1}{2}+i\frac{\sqrt{3}}{2}\right)
\right) \left( x-3\left( \frac{-1}{2}-i\frac{\sqrt{3}}{2}\right) \right) 
\end{equation*}
Note also $\left( x-3\left( \frac{-1}{2}+i\frac{\sqrt{3}}{2}\right) \right)
\left( x-3\left( \frac{-1}{2}-i\frac{\sqrt{3}}{2}\right) \right)
=\allowbreak x^{2}+3x+9$ and so
\begin{equation*}
x^{3}-27=\left( x-3\right) \left( x^{2}+3x+9\right)
\end{equation*}
where the quadratic polynomial $x^{2}+3x+9$ cannot be factored without using
complex numbers.
\end{solution}

Note that even though the polynomial $x^{3}-27$ has all real coefficients,
it has some complex zeros, $3\left( \displaystyle
\frac{-1}{2}+i\displaystyle\frac{\sqrt{3}}{2}\right) ,$ and $3\left( 
\displaystyle\frac{-1}{2}-i\displaystyle\frac{\sqrt{3}}{2}\right) $. 
These zeros are complex conjugates of each other. It is always the case that if a polynomial has real 
coefficients and a complex root, it will also
have a root equal to the complex conjugate.