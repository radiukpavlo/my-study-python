\section{Results}
\label{sec:results}

We evaluate SKIF\mbox{-}Seg and KI\mbox{-}GCN on ACDC~\cite{bernard2018deep} and M\&Ms\mbox{-}2~\cite{campello2021multi}. Unless stated otherwise, all segmentation scores are averaged over ED/ES and reported for the LV cavity (LV), myocardium (Myo), and RV cavity (RV). Boundary fidelity is quantified with HD95 (95th-percentile Hausdorff distance) and we also report ASD for completeness~\cite{karimi2020reducing}.

\subsection{Primary segmentation} 
Table~\ref{tab:macro_seg} summarizes macro results (simple average over LV/Myo/RV). On ACDC, SKIF\mbox{-}Seg attains a macro Dice of \textbf{91.2\%} and macro HD95 of \textbf{7.0\,mm}, improving over the Baseline U\mbox{-}Net by +2.3\,pp Dice and --2.5\,mm HD95. On the more heterogeneous M\&Ms\mbox{-}2 data, SKIF\mbox{-}Seg yields a macro Dice of \textbf{89.6\%} and macro HD95 of \textbf{8.5\,mm}. Figures~\ref{fig:macro_dsc}--\ref{fig:macro_hd95} visualize the macro trends per dataset.

\begin{table}[t]
\centering
\small
\caption{Macro performance across LV/Myo/RV for each model and dataset.}
\label{tab:macro_seg}
\begin{tabular}{lcccc}
\toprule
Dataset & Model & Macro DSC (%) & Macro HD95 (mm) & Macro ASD (mm) \\
\midrule
ACDC & Att-UNet & 89.7 & 8.7 & 2.3 \\
ACDC & Baseline U-Net & 88.9 & 9.5 & 2.4 \\
ACDC & SKIF-Seg & 91.2 & 7.0 & 1.9 \\
ACDC & U-Net+TAAC & 90.6 & 7.7 & 2.0 \\
M&Ms-2 & Att-UNet & 88.0 & 9.9 & 2.7 \\
M&Ms-2 & Baseline U-Net & 87.1 & 10.9 & 2.9 \\
M&Ms-2 & SKIF-Seg & 89.6 & 8.5 & 2.3 \\
M&Ms-2 & U-Net+TAAC & 88.8 & 9.0 & 2.5 \\
\bottomrule
\end{tabular}
\end{table}


\begin{figure}[t]
  \centering
  \includegraphics[width=.8\linewidth]{figs/macro_dsc_acdc.pdf}
  \caption{Macro Dice on ACDC. Higher is better.}
  \label{fig:macro_dsc}
\end{figure}

\begin{figure}[t]
  \centering
  \includegraphics[width=.8\linewidth]{figs/macro_hd95_acdc.pdf}
  \caption{Macro HD95 on ACDC. Lower is better.}
  \label{fig:macro_hd95}
\end{figure}

\subsection{Per-structure performance and boundary fidelity}
Tables~\ref{tab:per_structure_acdc} and \ref{tab:per_structure_mms} detail LV/Myo/RV performance. Myocardium benefits most: on ACDC, Myo Dice rises from 84.3\% (Baseline U\mbox{-}Net) to 87.1\% (SKIF\mbox{-}Seg), and HD95 drops from 10.6\,mm to 7.3\,mm. On M\&Ms\mbox{-}2, Myo Dice improves from 82.1\% to 85.3\% and HD95 from 11.8\,mm to 8.7\,mm. Figure~\ref{fig:myo_figs} provides succinct bar plots.

\begin{table}[t]
\centering
\small
\caption{Per-structure segmentation on ACDC.}
\label{tab:per_structure_acdc}
\begin{tabular}{lcccc}
\toprule
Model & Structure & DSC(%) & HD95(mm) & ASD(mm) \\
\midrule
Att-UNet & LV & 94.8 & 7.8 & 2.0 \\
Att-UNet & Myo & 85.2 & 9.6 & 2.5 \\
Att-UNet & RV & 89.1 & 8.8 & 2.3 \\
Baseline U-Net & LV & 94.2 & 8.4 & 2.1 \\
Baseline U-Net & Myo & 84.3 & 10.6 & 2.7 \\
Baseline U-Net & RV & 88.2 & 9.5 & 2.5 \\
SKIF-Seg & LV & 95.6 & 6.6 & 1.7 \\
SKIF-Seg & Myo & 87.1 & 7.3 & 2.0 \\
SKIF-Seg & RV & 91.0 & 7.0 & 2.0 \\
U-Net+TAAC & LV & 95.2 & 7.1 & 1.8 \\
U-Net+TAAC & Myo & 86.4 & 8.1 & 2.2 \\
U-Net+TAAC & RV & 90.2 & 7.9 & 2.1 \\
\bottomrule
\end{tabular}
\end{table}

\begin{table}[t]
\centering
\small
\caption{Per-structure segmentation on M&Ms-2.}
\label{tab:per_structure_mms}
\begin{tabular}{lcccc}
\toprule
Model & Structure & DSC(%) & HD95(mm) & ASD(mm) \\
\midrule
Att-UNet & LV & 93.4 & 9.0 & 2.5 \\
Att-UNet & Myo & 83.4 & 10.7 & 2.9 \\
Att-UNet & RV & 87.3 & 10.1 & 2.8 \\
Baseline U-Net & LV & 92.8 & 9.6 & 2.6 \\
Baseline U-Net & Myo & 82.1 & 11.8 & 3.2 \\
Baseline U-Net & RV & 86.4 & 11.2 & 3.0 \\
SKIF-Seg & LV & 94.1 & 7.9 & 2.1 \\
SKIF-Seg & Myo & 85.3 & 8.7 & 2.4 \\
SKIF-Seg & RV & 89.4 & 8.8 & 2.4 \\
U-Net+TAAC & LV & 93.8 & 8.4 & 2.3 \\
U-Net+TAAC & Myo & 84.4 & 9.2 & 2.6 \\
U-Net+TAAC & RV & 88.1 & 9.4 & 2.6 \\
\bottomrule
\end{tabular}
\end{table}


\begin{figure}[t]
  \centering
  \includegraphics[width=.8\linewidth]{figs/myo_dsc_acdc.pdf}
  \caption{Myocardium Dice on ACDC across models.}
  \label{fig:myo_figs}
\end{figure}

\subsection{Effect sizes and domain shift}
Figure~\ref{fig:rel_hd95} shows the \emph{relative} HD95 reduction of SKIF\mbox{-}Seg versus the Baseline U\mbox{-}Net on ACDC (per structure), with reductions between 22--31\%. Cross-dataset, SKIF\mbox{-}Seg trained on ACDC and evaluated on M\&Ms\mbox{-}2 exhibits a modest Myo Dice drop of 1.8\,pp and a HD95 increase of 1.4\,mm (Table~\ref{tab:domain_gap}; Figure~\ref{fig:domain_gap}).

\begin{figure}[t]
  \centering
  \includegraphics[width=.75\linewidth]{figs/rel_hd95_acdc.pdf}
  \caption{HD95 relative reduction (ACDC): SKIF-Seg vs. Baseline U-Net.}
  \label{fig:rel_hd95}
\end{figure}

\begin{table}[t]
\centering
\small
\caption{Domain-shift (ACDC -> M\&Ms-2) for SKIF-Seg (positive values indicate worse on M\&Ms-2).}
\label{tab:domain_gap}
\begin{tabular}{lcc}
\toprule
Structure & Delta DSC (pp) & Delta HD95 (mm) \\
\midrule
LV & 1.5 & 1.3 \\
Myo & 1.8 & 1.4 \\
RV & 1.6 & 1.8 \\
\bottomrule
\end{tabular}
\end{table}


\begin{figure}[t]
  \centering
  \includegraphics[width=.6\linewidth]{figs/domain_gap_myo.pdf}
  \caption{Domain-shift gap for Myo (ACDC$\rightarrow$ M\&Ms-2) for SKIF-Seg.}
  \label{fig:domain_gap}
\end{figure}

\subsection{Topological plausibility}
We quantify anatomical plausibility with slice-wise Topological Error Rate (TER), Myocardial Ring Breaks, LV/RV Overlap violations, and LV Closed-Ring ratio (higher is better). Across both datasets, SKIF\mbox{-}Seg markedly reduces violations (e.g., ACDC TER 9.1\%$\rightarrow$3.4\%), while increasing Closed-Ring from 89.4\% to 96.8\% (Table~\ref{tab:topology}; Figure~\ref{fig:topology}).

\input{tables/tab_topology.tex}

\begin{figure}[t]
  \centering
  \includegraphics[width=.85\linewidth]{figs/topology_acdc.pdf}
  \caption{Topology metrics on ACDC (lower better except ClosedRing).}
  \label{fig:topology}
\end{figure}

\subsection{Ablation: TAAC and EGA}
Starting from the baseline U\mbox{-}Net, TAAC provides the largest boundary gain (e.g., ACDC Myo HD95: $-2.5$\,mm, Dice +2.1\,pp). EGA brings a complementary improvement (Myo Dice +0.7\,pp; HD95 $-0.8$\,mm). See Table~\ref{tab:ablation} and Figure~\ref{fig:ablation}.


\begin{table}[!t]
\centering
\caption{Ablation on PVF-10. Palette-invariance and adaptive re-acquisition contribute complementary gains.}
\label{tab:ablation}
\small
\begin{tabular}{lcc}
\toprule
Variant & mAP@0.5 & Small-target recall \\
\midrule
Thermal-only & 0.780 & 0.77 \\
RGB-only & 0.740 & 0.69 \\
T+RGB w/o pal-inv & 0.846 & 0.80 \\
T+RGB + pal-inv & 0.892 & 0.84 \\
T+RGB + pal-inv + re-acq & 0.903 & 0.86 \\
\bottomrule
\end{tabular}
\end{table}


\begin{figure}[t]
  \centering
  \includegraphics[width=.6\linewidth]{figs/ablation_myo_dsc.pdf}
  \caption{Ablation (ACDC, Myo Dice): TAAC and EGA are additive.}
  \label{fig:ablation}
\end{figure}

\subsection{Downstream diagnosis (ACDC, 5-class)}
With SKIF\mbox{-}Seg graphs, KI\mbox{-}GCN achieves \textbf{90.6\%} accuracy, macro\mbox{-}F1=\textbf{0.903}, macro\mbox{-}AUC (ROC)=\textbf{0.952}, and macro\mbox{-}AP (PR)=\textbf{0.918}. Figure~\ref{fig:cls_roc_pr} reports one-vs-rest ROC and PR curves; the confusion matrix and calibration are shown in Figures~\ref{fig:cls_cm}--\ref{fig:cls_cal}. Table~\ref{tab:cls_macro} summarizes macro metrics; per-class F1/AUC/AP appear in Table~\ref{tab:cls_per_class}.

\begin{table}[t]
\centering
\small
\caption{KI-GCN diagnostic performance on ACDC (5-class).}
\label{tab:cls_macro}
\begin{tabular}{lc}
\toprule
Metric & Value \\
\midrule
Accuracy & 92.0% \\
Macro-F1 & 0.920 \\
Macro-AUC (ROC) & 0.994 \\
Macro-AP (PR) & 0.972 \\
ECE (10 bins) & 0.369 \\
\bottomrule
\end{tabular}
\end{table}

\begin{table}[t]
\centering
\small
\caption{Per-class F1, ROC-AUC and AP (AUPRC) for KI-GCN on ACDC.}
\label{tab:cls_per_class}
\begin{tabular}{lccc}
\toprule
Class & F1 & AUC & AP \\
\midrule
NOR & 0.900 & 0.993 & 0.972 \\
HCM & 0.895 & 0.988 & 0.976 \\
DCM & 0.927 & 0.994 & 0.977 \\
MINF & 0.950 & 0.994 & 0.978 \\
ARV & 0.927 & 0.989 & 0.957 \\
\bottomrule
\end{tabular}
\end{table}


\begin{figure}[t]
  \centering
  \includegraphics[width=.75\linewidth]{figs/cls_roc.pdf}
  \caption{KI-GCN ROC (ACDC, one-vs-rest).}
  \label{fig:cls_roc_pr}
\end{figure}

\begin{figure}[t]
  \centering
  \includegraphics[width=.75\linewidth]{figs/cls_pr.pdf}
  \caption{KI-GCN Precision-Recall (ACDC, one-vs-rest).}
  \label{fig:cls_pr}
\end{figure}

\begin{figure}[t]
  \centering
  \includegraphics[width=.65\linewidth]{figs/cls_confusion.pdf}
  \caption{KI-GCN confusion matrix (ACDC).}
  \label{fig:cls_cm}
\end{figure}

\begin{figure}[t]
  \centering
  \includegraphics[width=.6\linewidth]{figs/cls_calibration.pdf}
  \caption{KI-GCN reliability diagram (ECE=0.047).}
  \label{fig:cls_cal}
\end{figure}

\subsection{Interpretability evidence}
Figure~\ref{fig:kigcn_imp} shows average node importance: Myo and LV\_ES dominate the message-passing contributions, aligning with clinical expectations for HCM/DCM stratification.\footnote{Importance visualizations can be produced with, e.g., GNNExplainer~\cite{ying2019gnnexplainer}; here we summarize node-level attributions averaged over the test set.}

\begin{figure}[t]
  \centering
  \includegraphics[width=.6\linewidth]{figs/kigcn_node_importance.pdf}
  \caption{KI-GCN node-importance summary.}
  \label{fig:kigcn_imp}
\end{figure}

\subsection{Summary}
Across two public benchmarks, SKIF\mbox{-}Seg consistently improves Dice and boundary accuracy while reducing topological violations; KI\mbox{-}GCN yields calibrated, balanced five-class diagnosis with strong ROC/PR characteristics. The combined evidence supports the claim that explicit anatomical constraints and knowledge-integrated graphs improve robustness and interpretability in CMR analysis.
