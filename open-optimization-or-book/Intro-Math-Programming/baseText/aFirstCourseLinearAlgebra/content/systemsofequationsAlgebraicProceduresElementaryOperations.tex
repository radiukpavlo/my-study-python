\subsection{Elementary Operations}

We begin this section with an example. Recall from Example \ref{exa:graphicalsoln} that the solution to the given system was $\left(x, y \right) = \left( -1, 4 \right)$. 

\begin{example}{Verifying an Ordered Pair is a Solution}{verifyingorderedpairsolution}
Algebraically verify that $\left(x, y \right) = \left( -1, 4 \right)$ is a solution to the following system of equations.
\[
\begin{array}{c}
x+y=3 \\
y-x=5  
%\label{eq1.2.1}
\end{array}
\]
\end{example}

\begin{solution} 
By graphing these two equations and identifying the point of
intersection, we previously found that $\left(x, y \right) = \left(
-1, 4 \right)$ is the unique solution.

We can verify algebraically by substituting these values
into the original equations, and ensuring that the equations hold. 
First, we substitute the values into the first equation and check that it equals $3$. 
\[
x + y = (-1)+(4) = 3
\]
This equals $3$ as needed, so we see that $\left( -1,4 \right)$ is a solution to the first equation. 
Substituting the values into the second equation yields
\[
y -x = (4) - (-1) = 4 + 1  = 5
\]
which is true.
For
$\left( x,y\right) =\left( -1,4\right) $ each equation is true and therefore, this is a solution to the system.
\end{solution}

Now, the interesting question is this:\ If you were not given
these numbers to verify, how could you algebraically determine the solution? Linear algebra gives us the tools needed to answer this
question. The following basic operations are important tools that we will  utilize.

\begin{definition}{Elementary Operations}{elementaryoperations}
\textbf{Elementary operations} \index{elementary operations} are those
operations consisting of the following.

\begin{enumerate}
\item Interchange the order in which the equations are listed.

\item Multiply any equation by a nonzero number.

\item Replace any equation with itself added to a multiple of another
equation.

\end{enumerate}
\end{definition}

It is important to note that none of these operations will 
change the set of solutions of the system of equations. In fact, elementary operations are the {\em key tool \em}we use in linear algebra to find solutions to systems of equations. 

Consider the following example.  

\begin{example}{Effects of an Elementary Operation}{effectsofelementaryoperation}
Show that the system 
\begin{equation*}
\begin{array}{c}
x+y=7 \\
2x-y=8
\end{array}
\end{equation*}
has the same solution as the system 
\begin{equation*}
\begin{array}{c}
x+y=7 \\
-3y=-6
\end{array}
\end{equation*}
\end{example}

\begin{solution}
Notice that the second system has been obtained by taking the second equation of the first system
and adding -2 times the first equation, as follows:
\begin{equation*}
2x-y + (-2)(x+y) = 8 + (-2)(7)
\end{equation*}
By simplifying, we obtain
\begin{equation*}
-3y=-6
\end{equation*}
which is the second equation in the second system.
Now, from here we can solve for $y$ and see that $y=2$. Next, we substitute this value into the first equation as follows \:
\begin{equation*}
x+y=x+2=7
\end{equation*}
Hence $x=5$ and so $\left( x,y\right) = \left(5,2 \right)$ is a solution to the second system.  
We want to check if $\left(5,2 \right)$ is also a solution to the first system. We check this by substituting $\left(x, y \right) = \left(5,2 \right)$
into the system and ensuring the equations are true.
\begin{equation*}
\begin{array}{c}
x+y = \left(5 \right)+ \left( 2 \right) = 7 \\
2x-y= 2 \left(5 \right) - \left( 2 \right) = 8
\end{array}
\end{equation*}
Hence, $\left(5,2 \right)$ is also a solution to the first system. 
\end{solution}

This example illustrates how an elementary operation applied to a system of two equations in two variables
does not affect the solution set. However, a linear system may involve many equations and many variables and there is no 
reason to limit our study to small systems. For any size of system in any number of variables, the
solution set is still the collection of solutions to the equations. In every
case, the above operations of Definition \ref{def:elementaryoperations} do not
change the set of solutions to the system of linear equations.

In the following theorem, we use the notation $E_i$ to represent an equation, while $b_i$ denotes a
constant. 

\begin{theorem}{Elementary Operations and Solutions}{elementaryoperationsandsolns}
Suppose you have a system of two linear equations 
\begin{equation}
 \begin{array}{c}
  E_{1}=b_{1}\\
  E_{2}=b_{2}
\end{array} \label{system}
\end{equation}
Then the following systems have the same solution set as \ref{system}: 
\begin{enumerate}
\item   \begin{equation}
	\begin{array}{c}
	E_{2}=b_{2}\\
	E_{1}=b_{1}
	\end{array}
	\label{thm1.9.1}
	\end{equation}
\item  \begin{equation}
	\begin{array}{c}
	E_{1}=b_{1} \\
	kE_{2}=kb_{2}\\        
	\end{array}
	\label{thm1.9.2}
	\end{equation}
  for any scalar $k$, provided $k\neq0$.
\item \begin{equation}
      \begin{array}{c}
       E_{1}=b_{1} \\
       E_{2}+kE_{1}=b_{2}+kb_{1}
       \end{array}  
	\label{thm1.9.3}
	\end{equation}
	for any scalar  $k$ (including $k=0$).

\end{enumerate}
\end{theorem}

Before we proceed with the proof of Theorem \ref{thm:elementaryoperationsandsolns}, 
let us consider this theorem in context of Example \ref{exa:effectsofelementaryoperation}. Then,
\begin{equation*}
\begin{array}{cc}
E_{1} = x+y, & b_{1} = 7 \\
E_{2} = 2x-y, & b_{2} = 8 
\end{array}
\end{equation*}
Recall the elementary operations that we used to modify the system in the solution to the example. 
First, we added $\left( -2 \right)$ times the first equation to the second equation.
In terms of Theorem \ref{thm:elementaryoperationsandsolns}, this action is given by
\begin{equation*}
E_{2} + \left( -2 \right) E_{1} = b_{2} + \left( -2 \right)b_{1}
\end{equation*}
or
\begin{equation*}
2x-y + \left( -2 \right) \left(x+y \right) = 8 + \left( -2 \right) 7
\end{equation*}
This gave us the second system in Example \ref{exa:effectsofelementaryoperation}, given by 
\begin{equation*}
\begin{array}{c}
E_{1} = b_{1} \\
E_{2} + \left( -2 \right) E_{1} = b_{2} + \left( -2 \right) b_{1}
\end{array}
\end{equation*}

From this point, we were able to 
find the solution to the system. Theorem \ref{thm:elementaryoperationsandsolns} tells us that the solution we 
found is in fact a solution to the original system.

\ifdefined\showproofs
We will now prove Theorem \ref{thm:elementaryoperationsandsolns}.

\begin{proof} 
\begin{enumerate}
\item
The proof that the systems \ref{system} and \ref{thm1.9.1} have the
same solution set is as follows. Suppose that $\left( x_{1},\cdots
,x_{n}\right) $ is a solution to $E_{1}=b_{1},E_{2}=b_{2}$. We want to
show that this is a solution to the system in \ref{thm1.9.1} above.
This is clear, because the system in \ref{thm1.9.1} is the original
system, but listed in a different order. Changing the order does not
effect the solution set, so $\left( x_{1},\cdots ,x_{n}\right) $ is a
solution to \ref{thm1.9.1}.

\item
Next we want to prove that the systems \ref{system} and \ref{thm1.9.2} have the
same solution set. That is  $E_{1}=b_{1},E_{2}=b_{2}$ has
the same solution set as the system $E_{1}=b_{1},kE_{2}=kb_{2}$ provided $k\neq
0 $. Let $\left( x_{1},\cdots ,x_{n}\right) $ be a
solution of $E_{1}=b_{1},E_{2}=b_{2},$. We want to show that it is a solution to $E_{1}=b_{1},kE_{2}=kb_{2}$.
Notice that the only difference between these two systems is that the second involves
multiplying the equation, $E_{2}=b_{2}$ by the scalar $k$. Recall that when you multiply both sides of an 
equation by the same number, the sides are still equal to each other. Hence if  $\left( x_{1},\cdots ,x_{n}\right) $
is a solution to $E_{2}=b_{2}$, then it will also be a solution to $kE_{2}=kb_{2}$. Hence, $\left( x_{1},\cdots ,x_{n}\right) $ is also
a solution to \ref{thm1.9.2}. 

Similarly, let $\left( x_{1},\cdots
,x_{n}\right) $ be a solution of $E_{1}=b_{1},kE_{2}=kb_{2}$. Then we can 
multiply the equation $kE_{2}=kb_{2}$ by the scalar $1/k$, which is possible only because we have required that $k\neq
0$. Just as above, this action preserves equality and we obtain the equation $E_{2} = b_{2}$. 
Hence $\left( x_{1},\cdots ,x_{n}\right)$ is also a solution to  $E_{1}=b_{1},E_{2}=b_{2}.$ 

\item
Finally, we will prove that the systems \ref{system} and
\ref{thm1.9.3} have the same solution set. We will show that any
solution of $E_{1}=b_{1},E_{2}=b_{2}$ is also a solution of
\ref{thm1.9.3}. Then, we will show that any solution of \ref{thm1.9.3}
is also a solution of $E_{1}=b_{1},E_{2}=b_{2}$.  Let $\left(
x_{1},\cdots ,x_{n}\right) $ be a solution to
$E_{1}=b_{1},E_{2}=b_{2}$. Then in particular it solves $E_{1} =
b_{1}$. Hence, it solves the first equation in \ref{thm1.9.3}.
Similarly, it also solves $E_{2} = b_{2}$. By our proof of
\ref{thm1.9.2}, it also solves $kE_{1}=kb_{1}$.  Notice that if we add
$E_{2}$ and $kE_{1}$, this is equal to $b_{2} + kb_{1}$. Therefore, if
$\left( x_{1},\cdots ,x_{n}\right) $ solves $E_{1}=b_{1},E_{2}=b_{2}$
it must also solve $E_{2}+kE_{1}=b_{2}+kb_{1}$.

Now suppose $\left( x_{1},\cdots ,x_{n}\right)$  solves the system 
$E_{1}=b_{1}, E_{2}+kE_{1}=b_{2}+kb_{1}$. Then in particular it is a solution of $E_{1} = b_{1}$. Again by our proof of \ref{thm1.9.2},
 it is also a solution to $kE_{1}=kb_{1}$. Now if we subtract these equal quantities from both sides of 
$E_{2}+kE_{1}=b_{2}+kb_{1}$ we obtain $E_{2}=b_{2}$, which shows that the solution also satisfies
$E_{1}=b_{1},E_{2}=b_{2}.$ 
\end{enumerate}
\end{proof}
\fi
Stated simply, the above theorem shows that the elementary operations do not
change the solution set of a system of equations.

We will now look at an example of a system of three equations and three variables.
Similarly to the previous examples, the goal is to find values for $x,y,z$ such that each of the given equations
are satisfied when these values are substituted in.

\begin{example}{ Solving a System of Equations with Elementary Operations}{solvingasystemwithelementaryops}
Find the solutions to the system,

\begin{equation}
\begin{array}{c}
x+3y+6z=25 \\
2x+7y+14z=58 \\
2y+5z=19
\end{array}
\label{solvingasystem1}
\end{equation}
\end{example}

\begin{solution}
We can relate this system to Theorem \ref{thm:elementaryoperationsandsolns} above. In this case, we have
\begin{equation*}
\begin{array}{c c}
E_{1} = x + 3y + 6z, & b_{1} = 25\\
E_{2} = 2x+7y+14z, & b_{2} = 58 \\
E_{3} = 2y+5z, & b_{3} = 19 
\end{array}
\end{equation*}
Theorem \ref{thm:elementaryoperationsandsolns} claims that if we do elementary operations on this system, 
we will not change the solution set. Therefore,
we can solve this system using the elementary operations given in Definition \ref{def:elementaryoperations}.
First, replace the second equation by $\left( -2\right)$
times the first equation added to the second. This yields the system
\begin{equation}
\begin{array}{c}
x+3y+6z=25 \\
y+2z=8 \\
2y+5z=19
\end{array}
\label{solvingasystem2}
\end{equation}
Now, replace the third equation with $\left( -2\right) $ times the
second added to the third. This yields the system
\begin{equation}
\begin{array}{c}
x+3y+6z=25 \\
y+2z=8 \\
z=3
\end{array}
\label{solvingasystem3}
\end{equation}
At this point, we can easily find the solution. Simply take $z=3$ and substitute this back into the
 previous equation to solve for $y$, and similarly to solve for $x$.
\begin{equation*}
\begin{array}{c}
x + 3y + 6 \left(3 \right) = x + 3y + 18 = 25\\
y + 2 \left(3 \right) = y + 6 = 8 \\
z = 3
\end{array}
\end{equation*}
The second equation is now
\begin{equation*}
y+6=8
\end{equation*}
You can see from this equation that $y = 2$. Therefore, we can substitute this value into the first equation as follows:
\begin{equation*}
x + 3 \left(2 \right) + 18 = 25
\end{equation*}
By simplifying this equation, we find that $x=1$. 
Hence, the solution to this system is $\left( x,y,z \right) = \left( 1,2,3 \right)$. 
This process is called \textbf{back substitution}. \index{back substitution}

Alternatively, in \ref{solvingasystem3} you could have continued as follows. Add $\left(
-2\right) $ times the third equation to the second and then add $\left(
-6\right) $ times the second to the first. This yields
\begin{equation*}
\allowbreak
\begin{array}{c}
x+3y=7 \\
y=2 \\
z=3
\end{array}
\end{equation*}
Now add $\left( -3\right) $ times the second to the first. This yields
\begin{equation*}
\allowbreak
\begin{array}{c}
x=1 \\
y=2 \\
z=3
\end{array}
\end{equation*}
a system which has the same solution set as the original system. This
avoided back substitution and led to the same solution set. It is your decision which you prefer to use, as both methods lead to the correct solution,
$\left( x,y,z \right) = \left(1,2,3\right)$.
\end{solution}