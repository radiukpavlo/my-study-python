\subsection{Gaussian Elimination}

The work we did in the previous section will always find the solution to the system. In this section, we will explore a less cumbersome way to 
find the solutions. First, we will represent a linear system with an \textbf{augmented matrix.} \index{matrix!augmented matrix}
A \textbf{matrix} \index{matrix} is simply a rectangular array of numbers. The size or dimension of a matrix \index{matrix!dimension} is defined as $m\times n$ where $m$ is the
number of rows and $n$ is the number of columns. In order to construct an augmented matrix
from a linear system, we create a \textbf{coefficient matrix} \index{matrix!coefficient matrix}from the coefficients of the variables in the system, as well as a
\textbf{constant matrix} from the constants. The coefficients from one equation of the system create one row of the augmented matrix. 

For example, consider the linear system in Example \ref{exa:solvingasystemwithelementaryops}
\begin{equation*}
\begin{array}{c}
x+3y+6z=25 \\
2x+7y+14z=58 \\
2y+5z=19
\end{array}
\end{equation*}
This system can be written as an augmented matrix, as follows \:
\begin{equation*}
\leftB 
\begin{array}{rrr|r}
1 & 3 & 6 & 25 \\
2 & 7 & 14 & 58 \\
0 & 2 & 5 & 19
\end{array}
\rightB 
\end{equation*}

Notice that it has exactly the same information as the original system. Here it is
understood that the first column contains the coefficients from $x$ in each equation, in order,   
$\leftB
\begin{array}{r}
1 \\
2 \\
0
\end{array}
\rightB .$ Similarly, we create a column from the coefficients on $y$ in each equation, $\leftB
\begin{array}{r}
3 \\
7 \\
2
\end{array}
\rightB $ and a column from the coefficients on $z$ in each equation, $\leftB
\begin{array}{r}
6 \\
14 \\
5
\end{array}
\rightB .$ For a system of more than three variables, we would continue in this way constructing a column for each variable.
Similarly, for a system of less than three variables, we simply construct a column for each variable. 

Finally, we construct a column from the constants of the equations,
$\leftB
\begin{array}{r}
25\\
58\\
19
\end{array}
\rightB .$ 

The rows of the augmented matrix correspond to the equations in the system. For example, the top
row in the augmented matrix, 
$\leftB
\begin{array}{rrrrr}
1 & 3 & 6 & | & 25
\end{array}
\rightB$
 corresponds to the equation
\begin{equation*}
x+3y+6z=25.
\end{equation*}

Consider the following definition. 

\begin{definition}{Augmented Matrix of a Linear System}{augmentedmatrix}
For a linear system of the form
\begin{equation*}
\begin{array}{c}
a_{11}x_{1}+\cdots +a_{1n}x_{n}=b_{1} \\
\vdots \\
a_{m1}x_{1}+\cdots +a_{mn}x_{n}=b_{m}
\end{array}
\end{equation*}
where the $x_{i}$ are variables and the $a_{ij}$ and $b_{i}$ are constants, 
the augmented matrix \index{matrix!augmented matrix} of this system is given by 
\begin{equation*}
\leftB
\begin{array}{rrr|r}
a_{11} & \cdots & a_{1n} &  b_{1} \\
\vdots &  & \vdots &  \vdots \\
a_{m1} & \cdots & a_{mn} &  b_{m}
\end{array}
\rightB 
\end{equation*}
\end{definition}

Now, consider elementary operations in the context of the augmented matrix. The elementary operations in Definition \ref{def:elementaryoperations}
can be used on the rows just as we used them on equations previously. Changes to a system of equations in as a result of an
elementary operation are equivalent to changes in the augmented matrix resulting from the corresponding row operation. Note that Theorem \ref{thm:elementaryoperationsandsolns} implies that
any elementary row operations used on an augmented matrix will not change the solution to the corresponding system of equations.
We now formally define elementary row operations. These are the {\em key tool \em}we will use to find solutions to systems of equations. 

\begin{definition}{Elementary Row Operations}{rowoperations}
The \textbf{elementary row operations} (also known as \textbf{row operations}) consist of the following \index{row operations} \index{elementary row operations}

\begin{enumerate}
\item Switch two rows.

\item Multiply a row by a nonzero number.


\item Replace a row by any multiple of another row added to it.
\end{enumerate}
\end{definition}

Recall how we solved Example \ref{exa:solvingasystemwithelementaryops}. 
We can do the exact same steps as above, except now in the context of an augmented matrix and using row operations. 
The augmented matrix of this system is 
\begin{equation*}
\leftB 
\begin{array}{rrr|r}
1 & 3 & 6 &  25 \\
2 & 7 & 14 &  58 \\
0 & 2 & 5 &  19
\end{array}
\rightB 
\end{equation*}
Thus the first step in
solving the system given by \ref{solvingasystem1} would be to take $\left( -2\right) $ times the first row
of the augmented matrix and add it to the second row,
\begin{equation*}
\leftB
\begin{array}{rrr|r}
1 & 3 & 6 & 25 \\
0 & 1 & 2 & 8 \\
0 & 2 & 5 & 19
\end{array}
\rightB 
\end{equation*}
Note how this corresponds to \ref{solvingasystem2}. Next take $\left( -2\right) $
times the second row and add to the third,
\begin{equation*}
\leftB 
\begin{array}{rrr|r}
1 & 3 & 6 & 25 \\
0 & 1 & 2 & 8 \\
0 & 0 & 1 & 3
\end{array}
\rightB
\end{equation*}
This augmented matrix corresponds to the system
\begin{equation*}
\begin{array}{c}
x+3y+6z=25 \\
y+2z=8 \\
z=3
\end{array}
\end{equation*}
which is the same as \ref{solvingasystem3}. By back substitution you obtain the solution
$x=1,y=2,$ and $z=3.$

Through a systematic
procedure of row operations, we can simplify an augmented matrix and carry it to \textbf{\ef} or \textbf{\rref}, which we define next. These 
forms are used to find the solutions of the system of equations corresponding to the augmented matrix.

In the following definitions, the term
\textbf{leading entry}
\index{leading entry}refers to
 the first nonzero entry of a row when scanning the row
from left to right.

\begin{definition}{\EF}{echelonform}
An augmented matrix is
\index{\ef} in \textbf{\ef} if

\begin{enumerate}
\item All nonzero rows are above any rows of zeros.

\item Each leading entry of a row is in a column to the right of the leading entries of any row above it. 

%%Definition using original Kuttler definition ends here

%%Definition using Style B definition of \ref
\item Each leading entry of a row is equal to $1$.

\end{enumerate}
\end{definition}

%%For Style B:
We also consider another reduced form of the augmented matrix which has one further condition. 

%%For Original Kuttler:
%% We also consider another reduced form of the augmented matrix which has further conditions.

\begin{definition}{\RREF}{rref}
An augmented 
\index{\rref} matrix is in \textbf{\rref } if

\begin{enumerate}
\item All nonzero rows are above any rows of zeros.

\item Each leading entry of a row is in a column to the right of the leading
entries of any rows above it.

\item Each leading entry of a row is equal to $1$.

\item All entries in a column above and below a leading entry are zero.

\end{enumerate}
\end{definition}

%%For Style B definition:
Notice that the first three conditions on a \rref  \; matrix are the same as those for \ef. 


%%For original Kuttler definition:
%%Notice that the first two conditions on a \rref  \; matrix are the same as those for \ef.

Hence, every \rref \;matrix is also in \ef. The converse is not necessarily true;
 we cannot assume that every matrix in \ef \;is also in \rref. However, it often happens
that the \ef \;is sufficient to provide information about the solution of a system.

The following examples describe matrices in these various forms. As an exercise, take the time
to carefully verify that they are in the specified form.

\begin{example}{Not in \EF}{matricesnotechelon}
The following augmented matrices are not in \ef \;(and therefore also not in \rref).

\begin{equation*}
\leftB
\begin{array}{rrr|r}
0 & 0 & 0 & 0 \\
1 & 2 & 3 & 3 \\
0 & 1 & 0 & 2 \\
0 & 0 & 0 & 1 \\
0 & 0 & 0 & 0
\end{array}
\rightB ,\leftB
\begin{array}{rr|r}
1 & 2 & 3 \\
2 & 4 & -6 \\
4 & 0 & 7
\end{array}
\rightB ,\leftB
\begin{array}{rrr|r}
0 & 2 & 3 & 3 \\
1 & 5 & 0 & 2 \\
7 & 5 & 0 & 1 \\
0 & 0 & 1 & 0
\end{array}
\rightB 
\end{equation*}
\end{example}

\begin{example}{Matrices in \EF}{matricesnotrref}

%%For Original Kuttler definition, can change these matrices to not have leading 1s.

The following augmented matrices are in \ef, but not in \rref.
\begin{equation*}
\leftB
\begin{array}{rrrrr|r}
1 & 0 & 6 & 5 & 8 & 2 \\
0 & 0 & 1 & 2 & 7 & 3 \\
0 & 0 & 0 & 0 & 1 & 1 \\
0 & 0 & 0 & 0 & 0 & 0
\end{array}
\rightB ,\leftB
\begin{array}{rrr|r}
1 & 3 & 5 & 4 \\
0 & 1 & 0 & 7 \\
0 & 0 & 1 & 0 \\
0 & 0 & 0 & 1 \\
0 & 0 & 0 & 0
\end{array}
\rightB, \leftB
\begin{array}{rrr|r}
1 & 0 & 6 & 0 \\
0 & 1 & 4 & 0 \\
0 & 0 & 1 & 0 \\
0 & 0 & 0 & 0
\end{array}
\rightB
\end{equation*}
\end{example}

Notice that we could apply further row operations to these matrices to carry them to \rref. Take the time to try that on your own. 
Consider the following matrices, which are in \rref. 

\begin{example}{Matrices in\RREF}{rrefmatrices}
The following augmented matrices are in \rref.
\begin{equation*}
\leftB
\begin{array}{rrrrr|r}
1 & 0 & 0 & 5 & 0 & 0 \\
0 & 0 & 1 & 2 & 0 & 0 \\
0 & 0 & 0 & 0 & 1 & 1 \\
0 & 0 & 0 & 0 & 0 & 0
\end{array}
\rightB ,\leftB
\begin{array}{rrr|r}
1 & 0 & 0 & 0 \\
0 & 1 & 0 & 0 \\
0 & 0 & 1 & 0 \\
0 & 0 & 0 & 1 \\
0 & 0 & 0 & 0
\end{array}
\rightB , \leftB
\begin{array}{rrr|r}
1 & 0 & 0 & 4 \\
0 & 1 & 0 & 3 \\
0 & 0 & 1 & 2 
\end{array}
\rightB
\end{equation*}
\end{example}

One way in which the \ef \;of a matrix is useful is in identifying the pivot positions and pivot columns of the matrix.

\begin{definition}{Pivot Position and Pivot Column}{pivotpositionandcolumn}
A \textbf{pivot position}
\index{pivot position}in a matrix is the location of a leading entry in the \ef of a matrix.
A \textbf{pivot column}
\index{pivot column} is a column that contains a pivot position.
\end{definition}

For example consider the following.

\begin{example}{Pivot Position}{pivotposition}
Let
\begin{equation*}
A=\leftB
\begin{array}{rrr|r}
1 & 2 & 3 & 4 \\
3 & 2 & 1 & 6 \\
4 & 4 & 4 & 10
\end{array}
\rightB
\end{equation*}
Where are the pivot positions
and pivot columns of the augmented matrix $A$?
\end{example}

\begin{solution} 
%%Solution as per Style B definition of row-echelon form
The \ef \;of this matrix is
\begin{equation*}
\leftB
\begin{array}{rrr|r}
1 & 2 & 3 & 4 \\
0 & 1 & 2 & \vspace{0.05in}\frac{3}{2} \\
0 & 0 & 0 & 0
\end{array}
\rightB
\end{equation*}


%%Solution as per Kuttler's original definition of echelon form
%% The \ef \;of this matrix is
%%\begin{equation*}
%%\leftB
%%\begin{array}{rrr|r}
%%1 & 2 & 3 & 4 \\
%%0 & -4& -8 & -6 \\
%%0 & 0 & 0 & 0
%%\end{array}
%%\rightB
%%\end{equation*} 


This is all we need in this example, but note that this matrix is not in \rref.

In order to identify the pivot positions in the original matrix, we look for the leading entries in the \ef \hspace{0.5mm} of the matrix. 
Here, the entry in the first row and first column, as well as the entry in the second row and second column 
are the leading entries. Hence, these locations are the pivot positions. We identify the pivot positions in the original matrix, 
as in the following:\
\begin{equation*}
\leftB
\begin{array}{rrr|r}
\fbox{1} & 2 & 3 & 4 \\
3 & \fbox{2} & 1 & 6 \\
4 & 4 & 4 & 10
\end{array}
\rightB
\end{equation*}
Thus the pivot columns in the matrix are the first two columns. 
\end{solution}

The following is an algorithm for carrying a matrix
to \ef \;and \rref. You may wish to use this algorithm to carry the above matrix to \ef \;or \rref \;yourself for practice. 

\begin{algorithm}{\RREF Algorithm}{rrefalgorithm}

This algorithm provides a method for using row operations to take a matrix to 
its \rref \index{\rref!algorithm}.
We begin with the matrix in its original form. 

\begin{enumerate}
\item Starting from the left, find the first nonzero column. This is the first pivot
column, and the position at the top of this column is the first pivot
position. Switch rows if necessary to place a nonzero number in the first
pivot position.

\item Use row operations to make the entries below the first pivot
position (in the first pivot column) equal to zero.

%%Algorithm for the Style B definition of row-echelon form
\item  Ignoring the row containing the first pivot position, repeat steps 1 and 2 with the remaining rows. 
Repeat the process until there are no more rows
to modify. 

\item Divide each nonzero row by the value of the leading entry, so that the leading entry becomes $1$. The matrix will then be in \ef.

\suspend{enumerate}
The following step will carry the matrix from \ef \;to \rref. 

\resume{enumerate}

\item Moving from right to left, use row operations to create zeros in the entries of the pivot columns which are above the pivot positions. The result
will be a matrix in \rref.
\end{enumerate}

%%Algorithm for the original Kuttler definition of echelon form
%%\item Ignoring the row containing the first pivot position, repeat steps 1 and 2 with the remaining rows. Repeat the process until there 
%% are not more rows to modify. The matrix will then be in \ef.
%%\suspend{enumerate}
%%The following steps will carry the matrix from \ef \;to \rref.
%%\resume{enumerate}
%%\item Moving from right to left, use row operations to create zeros in the entries of the pivot columns which are above the pivot positions.
%%\item Divide each nonzero row by the value of the leading entry, so that the leading entry becomes $1$. The result will be 
%% a matrix in \rref.
%%\end{enumerate}
\end{algorithm}

Most often we will apply this algorithm to an augmented matrix in order to find the solution
to a system of linear equations. However, we can use this algorithm to compute the \rref \;of any matrix which could be useful in other applications.

Consider the following example of Algorithm \ref{algo:rrefalgorithm}.

\begin{example}{Finding \EF and \\ \RREF of a Matrix}{findingechelonandrref}
Let
\begin{equation*}
A = \leftB
\begin{array}{rrr}
0 & -5 & -4 \\
1 & 4 & 3 \\
5 & 10 & 7
\end{array}
\rightB
\end{equation*}
Find the \ef \;of $A$. Then complete
the process until $A$ is in \rref.
\end{example}

\begin{solution} In working through this example, we will use the steps outlined in Algorithm \ref{algo:rrefalgorithm}.
\begin{enumerate}

\item The first pivot column is the first column of the matrix, as this is the first nonzero column from the left. 
Hence the first pivot position is the one in the first row and first column. Switch the first two rows to 
obtain a nonzero entry in the first pivot position, outlined in a box below.
\begin{equation*}
\leftB
\begin{array}{rrr}
\fbox{1} & 4 & 3 \\
0 & -5 & -4 \\
5 & 10 & 7 
\end{array}
\rightB
\end{equation*}

\item Step two involves creating zeros in the entries below the first pivot position. The first entry of the second row
is already a zero. All we need to do is subtract $5$ times the first row from the third row. 
The resulting matrix is 
\begin{equation*}
\leftB
\begin{array}{rrr}
1 & 4 & 3 \\
0 & -5 & -4 \\
0 & 10 & 8 
\end{array}
\rightB
\end{equation*} 

%%Solution using Style B definition of \ef

\item Now ignore the top row.  Apply steps $1$ and $2$ to the smaller matrix
\begin{equation*}
\leftB
\begin{array}{rr}
-5 & -4\\
10 & 8 
\end{array}
\rightB
\end{equation*}
In this matrix, the first column is a pivot column, and $-5$ is in the first pivot position. Therefore, we need to create a zero below it.
To do this, add $2$ times the first row (of this matrix) to the second. The resulting matrix is
\begin{equation*}
\leftB
\begin{array}{rr}
-5 & -4\\
0 & 0
\end{array}
\rightB
\end{equation*}
Our original matrix now looks like
\begin{equation*}
\leftB
\begin{array}{rrr}
1 & 4 & 3 \\
0 & -5 & -4 \\
0 & 0 & 0 
\end{array}
\rightB
\end{equation*}
We can see that there are no more rows to modify.

\item Now, we need to create leading $1$s in each row. The first row already has a leading $1$ so no work is needed here.
Divide the second row by $-5$ to create a leading $1$. The resulting matrix is
\begin{equation*}
\leftB
\begin{array}{rrr}
1 & 4 & 3 \\
0 & 1 & \vspace{0.05in}\frac{4}{5} \\
0 & 0 & 0
\end{array}
\rightB
\end{equation*}
This matrix is now in \ef.

\item Now create zeros in the entries above pivot positions in each column, in order to carry this matrix all the way to \rref. 
Notice that there is no pivot position in the third column so we do not need to create any zeros in this column! The column in which we 
need to create zeros is the second. To do so, subtract $4$ times the second row from the first row. The resulting matrix is
\begin{equation*}
\leftB
\begin{array}{rrr}
1 & 0 & -\vspace{0.05in}\frac{1}{5} \\
0 & 1 & \vspace{0.05in}\frac{4}{5} \\
0 & 0 & 0
\end{array}
\rightB
\end{equation*}
\end{enumerate}
This matrix is now in \rref.
\end{solution}

%%Solution using the original Kuttler definition of \ef
%%\item Now ignore the top row.  Apply steps $1$ and $2$ to the smaller matrix
%%\begin{equation*}
%%\leftB
%%\begin{array}{rr}
%%-5 & -4\\
%%10 & 8 
%%\end{array}
%%\rightB.
%%\end{equation*}
%%In this matrix, $-5$ is in the pivot position. Therefore, we need to create a zero below it.
%%To do this, add $2$ times the first row (of this matrix) to the second. The resulting matrix is
%%\begin{equation*}
%%\leftB
%%\begin{array}{rr}
%%-5 & 4\\
%%0 & 0
%%\end{array}
%%\rightB.
%%\end{equation*}
%%Our original matrix now looks like
%%\begin{equation*}
%%\leftB
%%\begin{array}{rrr}
%%1 & 4 & 3 \\
%%0 & -5 & -4 \\
%%0 & 0 & 0 
%%\end{array}
%%\rightB
%%\end{equation*}
%%We can see that there are no more rows to modify. Therefore, this matrix is in \ef.

%%\item Now, we wish to continue carrying this matrix all the way to \rref.
%%First, we want to create zeros in the entries above pivot positions in each column. 
%%Notice that there is no pivot position in the third column so we do not need to create any zeros in this column! The column which we 
%%need to create zeros in is the second. To do so, add $\frac{4}{5}$ times the second row from the first row. The resulting matrix is
%%\begin{equation*}
%%\leftB
%%\begin{array}{rrr}
%%1 & 0 & -\vspace{0.05in}\frac{1}{5} \\
%%0 & -5 & -4 \\
%%0 & 0 & 0
%%\end{array}
%%\rightB
%%\end{equation*}

%%\item Finally, we need to create leading $1$s in each row. The first row already has a leading $1$ so no work is needed here.
%%Divide the second row by $-5$ to create a leading $1$. The resulting matrix is
%%\begin{equation*}
%%\leftB
%%\begin{array}{rrr}
%%1 & 0 & -\vspace{0.05in}\frac{1}{5} \\
%%0 & 1 & \vspace{0.05in}\frac{4}{5} \\
%%0 & 0 & 0
%%\end{array}
%%\rightB
%%\end{equation*}
%%\end{enumerate}
%%This matrix is now in \rref.
%% \end{solution}


The above algorithm gives you a simple way to obtain the \ef \;and \rref \;of a matrix.
The main idea is to do row operations in such a way as to end up with a
matrix in \ef \;or \rref. This process is important because the resulting matrix will 
allow you to describe the
solutions to the corresponding linear system of equations in a meaningful way. 

In the next example, we look at how to solve a system of equations using the corresponding augmented matrix.

\begin{example}{Finding the Solution to a System}{systemwithnosoln}
Give the complete solution to the following system of equations \:
\begin{equation*}
\begin{array}{c}
2x+4y-3z=-1\\
5x+10y-7z=-2\\
3x+6y+5z=9
\end{array}
\end{equation*}
\end{example}

\begin{solution} The augmented matrix for this system is
\begin{equation*}
\leftB
\begin{array}{rrr|r}
2 & 4  & -3 & -1 \\
5 & 10 & -7 & -2 \\
3 & 6  &  5 &  9
\end{array}
\rightB
\end{equation*}

In order to find the solution to this system, we wish to carry the augmented matrix to \rref.
We will do so using Algorithm \ref{algo:rrefalgorithm}. Notice that
the first column is nonzero, so this is our first pivot column. The first entry in the first row, $2$, is the
first leading entry and it is in the first pivot position. We will use row operations
to create zeros in the entries below the $2$.  First, replace the second row with $-5$ times the first row plus $2$ times the second row. This yields
\begin{equation*}
\leftB
\begin{array}{rrr|r}
2 & 4 & -3 & -1 \\
0 & 0 & 1 & 1 \\
3 & 6 & 5 & 9
\end{array}
\rightB
\end{equation*}
Now, replace the third row with $-3$ times the first
row plus to $2$ times the third row. This yields
\begin{equation*}
\leftB
\begin{array}{rrr|r}
2 & 4 & -3 &  -1 \\
0 & 0 & 1 & 1 \\
0 & 0 & 1 & 21
\end{array}
\rightB 
\end{equation*}
Now the entries in the first column below the pivot position are zeros. We now look for the second pivot column, which in this case is column three. 
Here, the $1$ in the second row and third column is in the pivot position. 
We need to do just one row operation to create a zero below the $1$. 

Taking $-1$ times the second row and
adding it to the third row yields
\begin{equation*}
\leftB
\begin{array}{rrr|r}
2 & 4 & -3 & -1 \\
0 & 0 &  1 & 1 \\
0 & 0 &  0 & 20
\end{array}
\rightB 
\end{equation*}

We could proceed with the algorithm to carry this matrix to \ef \; or \rref. However, remember that 
we are looking for the solutions to the system of equations. Take another look at the third row of the matrix.
Notice that it corresponds to 
the equation
\begin{equation*}
0x+0y+0z=20
\end{equation*}
There is no solution to this equation because for all $x,y,z$, the left side will equal $0$ and $0\neq 20.$ This shows there is no
solution to the given system of equations. In other words, this system is inconsistent.
\end{solution}

The following is another example of how to find the solution to a system of equations by carrying the corresponding
augmented matrix to \rref. 

\begin{example}{An Infinite Set of Solutions}{infinitesetofsoln}
Give the complete solution to the system of equations \:
\begin{equation}
\begin{array}{c}
3x-y-5z=9 \\
y-10z=0 \\
-2x+y=-6
\end{array}
\end{equation}
\end{example}

\begin{solution} The augmented matrix of this system is
\begin{equation*}
\leftB
\begin{array}{rrr|r}
 3 & -1 &  -5 & 9 \\
 0 &  1 & -10 & 0 \\
-2 &  1 &   0 & -6
\end{array}
\rightB
\end{equation*}
In order to find the solution to this system, we will carry the augmented matrix to \rref, using Algorithm \ref{algo:rrefalgorithm}.
The first column is the first pivot column. We want to use row operations to create zeros beneath the first entry in this column, which is 
in the first pivot position. 
Replace the third row with $2$ times the first row added to $3$ times the
third row. This gives
\begin{equation*}
\leftB
\begin{array}{rrr|r}
3 & -1 & -5 & 9 \\
0 & 1 & -10 & 0 \\
0 & 1 & -10 & 0
\end{array}
\rightB 
\end{equation*}
Now, we have created zeros beneath the $3$ in the first column, so we move on to the second pivot column (which is the second column) and repeat the procedure. 
Take $-1$ times the second row and add to the third row.
\begin{equation*}
\leftB
\begin{array}{rrr|r}
3 & -1 &  -5 & 9 \\
0 &  1 & -10 & 0 \\
0 &  0 &   0 & 0
\end{array}
\rightB
\end{equation*}
The entry below the pivot position in the second column is now a zero. Notice that we have no more pivot columns because we 
have only two leading entries. 


%%Solution using Style B definition of row-echelon form
At this stage, we also want the leading entries to be equal to one. To do so, divide the first row by $3$.
\begin{equation*}
\leftB
\begin{array}{rrr|r}
1 & -\vspace{0.05in}\frac{1}{3} &  -\vspace{0.05in}\frac{5}{3} & 3 \\
0 &  1 & -10 & 0 \\
0 &  0 &   0 & 0
\end{array}
\rightB
\end{equation*}

This matrix is now in \ef. 

Let's continue with row operations until the matrix is in \rref. 
This involves creating zeros above the pivot positions in each pivot column. 
This requires only one step, which is to add $ \frac{1}{3}$ times the second row to the first row.
\begin{equation*}
\leftB
\begin{array}{rrr|r}
1 & 0 & -5 & 3 \\
0 & 1 & -10 & 0 \\
0 & 0 & 0 & 0
\end{array}
\rightB 
\end{equation*}
%end of Kuttler specific solution 

%%Solution using original Kuttler definition of echelon form
%%This matrix is now in \ef. We want to continue with row operations
%%until the matrix is in \rref. Therefore, we can begin creating zeros above the
%%pivot positions in each pivot column. This requires only one step, which
%%is to take the second row and add it to the first row.
%%At this stage, we also want the leading entries to be equal to one. 
%%To do so, divide the first row by $3.$ These steps yield the matrix
%%\begin{equation*}
%%\leftB
%%\begin{array}{rrr|r}
%%1 & 0 & -5 & 3 \\
%%0 & 1 & -10 & 0 \\
%%0 & 0 & 0 & 0
%%\end{array}
%%\rightB 
%%\end{equation*}

This is in \rref, which you should verify using Definition \ref{def:rref}.
 The equations corresponding to this \rref \;are
\[
\begin{array}{c}
x - 5z=3 \\
y - 10z = 0 
\end{array}
\]
or
\[
\begin{array}{c}
x=3+5z \\
y = 10z
\end{array}
\]

Observe that $z$ is not restrained by any equation. In fact,
 $z$ can equal any number. For example, we can let $z = t$, where we can choose $t$ to be any number.
In this context $t$ is called a \textbf{parameter} \index{parameter}. Therefore, the solution
set of this system is 
\[
\begin{array}{c}
x=3+5t \\
y=10t \\
z=t
\end{array}
\]
where $t$ is arbitrary. The system has an infinite set of solutions which are given by these equations. For any value of $t$ we select,
$x, y,$ and $z$ will be given by the above equations. For example, if we choose $t=4$ then the corresponding solution would be 
\[
\begin{array}{c}
x = 3 + 5 (4) = 23\\
y = 10(4)=40 \\
z=4
\end{array}
\]
\end{solution}

In Example \ref{exa:infinitesetofsoln} the solution involved one parameter. It may happen that the solution
to a system involves more than one parameter, as shown in the following example.

\begin{example}{A Two Parameter Set of Solutions}{twoparametersetofsoln}
Find the solution to the system
\begin{equation*}
\begin{array}{c}
x+2y-z+w=3 \\
x+y-z+w=1 \\
x+3y-z+w=5
\end{array}
\end{equation*}
\end{example}

\begin{solution} The augmented matrix is
\begin{equation*}
\leftB
\begin{array}{rrrr|r}
1 & 2 & -1 & 1 & 3 \\
1 & 1 & -1 & 1 & 1 \\
1 & 3 & -1 & 1 & 5
\end{array}
\rightB 
\end{equation*}
We wish to carry this matrix to \ef. Here, we will outline the row operations used. However, make sure that you 
understand the steps in terms of Algorithm \ref{algo:rrefalgorithm}.

Take $-1$ times the first row and add to the second. Then take $-1$ times
the first row and add to the third. This yields
\begin{equation*}
\leftB
\begin{array}{rrrr|r}
1 & 2 & -1 & 1 & 3 \\
0 & -1 & 0 & 0 & -2 \\
0 & 1 & 0 & 0 & 2
\end{array}
\rightB
\end{equation*}

%%Solution using Style B definition of row-echelon form
Now add the second row to the third row and divide the second row by $-1$.
\begin{equation}
\leftB
\begin{array}{rrrr|r}
1 & 2 & -1 & 1 & 3 \\
0 & 1 & 0 & 0 & 2 \\
0 & 0 & 0 & 0 & 0
\end{array}
\rightB  \label{twoparameters1}
\end{equation}

%%Solution using original Kuttler definition of echelon form
%%Now add the second row to the third row.
%%\begin{equation}
%%\leftB
%%\begin{array}{rrrr|r}
%%1 & 2 & -1 & 1 & 3 \\
%%0 & -1 & 0 & 0 & -2 \\
%%0 & 0 & 0 & 0 & 0
%%\end{array}
%%\rightB  \label{twoparameters1}
%%\end{equation}


This matrix is in \ef \;and we can see that $x$ and $y$ 
correspond to pivot columns, while $z$ and $w$ do not. Therefore, we will assign parameters to the variables $z$ and $w$.
Assign the parameter $s$ to $z$ and the parameter 
$t$ to $w.$ Then the first row yields the equation $x+2y-s+t=3$, while the second row yields the 
equation $y=2$. Since $y=2$, the first equation becomes  $x+4-s+t=3$
showing that the solution is given by
\begin{equation*}
\begin{array}{c}
x=-1+s-t \\
y=2 \\
z=s \\
w=t
\end{array}
\end{equation*}
It is customary to write this solution in
the form
\begin{equation}
\leftB
\begin{array}{c}
x \\
y \\
z \\
w
\end{array}
\rightB =\leftB
\begin{array}{c}
-1+s-t \\
2 \\
s \\
t
\end{array}
\rightB   \label{twoparameters2}
\end{equation}

\end{solution}

This example shows a system of equations with an infinite solution set which depends on two parameters. 
It can be less confusing in the case of an infinite solution set to first
place the augmented matrix in \rref \;rather than just
\ef \;before seeking to write down the description of the solution.

In the above steps, this means we don't stop with the \ef \;in equation \ref{twoparameters1}.
Instead we first place it in \rref \;as follows.
\begin{equation*}
\leftB
\begin{array}{rrrr|r}
1 & 0 & -1 & 1 & -1 \\
0 & 1 & 0 & 0 & 2 \\
0 & 0 & 0 & 0 & 0
\end{array}
\rightB 
\end{equation*}
Then the solution is $y=2$ from the second row and $x=-1+z-w$ from the
first. Thus letting $z=s$ and $w=t,$ the solution is given by \ref{twoparameters2}. 

You can see here that there are two paths to the correct answer, which both yield the
same answer. Hence, either approach may be used. The process which we first used in the above solution is called \textbf{Gaussian Elimination} 
\index{Gaussian Elimination}
This process involves carrying the matrix to \ef, converting back to equations, and using back substitution to find the solution. 
When you do row operations until you obtain \rref, the process is
called \textbf{Gauss-Jordan Elimination}. \index{Gauss-Jordan Elimination}

We have now found solutions for systems of equations with no solution and infinitely many solutions, with one parameter as well as two parameters. 
Recall the three types of solution sets which we discussed in the previous section; no solution, one solution, and infinitely many solutions. Each of these
types of solutions could be identified from the graph of the system. It turns out that we can also identify the type of solution
from the \rref \;of the augmented matrix.

\begin{itemize}
\item {\em No Solution: \em}
In the case where the system of equations has no solution, the \ef \; of the augmented matrix will have a row of the form 
\[
\leftB
\begin{array}{rrrrr}
0 & 0 & 0 & | & 1
\end{array}
\rightB
\]

This row indicates that the system is inconsistent and has no solution.

\item {\em One Solution: \em}
In the case where the system of equations has one solution, every column
of the coefficient matrix is a pivot column. 
The following is an example of an augmented matrix in \rref \hspace{0.5mm} for a system of equations with one solution. 
\begin{equation*}
\leftB
\begin{array}{rrr|r}
1 & 0 & 0 & 5 \\
0 & 1 & 0 & 0 \\ 
0 & 0 & 1 & 2 
\end{array}
\rightB
\end{equation*}

\item {\em Infinitely Many Solutions: \em} 
In the case where the system of equations has infinitely many solutions, the solution contains parameters. There
will be columns of the coefficient matrix which are not pivot columns.
The following are examples of augmented matrices in \rref \;for systems of equations with infinitely many solutions. 
\begin{equation*}
\leftB
\begin{array}{rrr|r}
1 & 0 & 0 & 5 \\
0 & 1 & 2 & -3 \\ 
0 & 0 & 0 & 0 
\end{array}
\rightB
\end{equation*}
or
\begin{equation*}
\leftB
\begin{array}{rrr|r}
1 & 0 & 0 & 5 \\
0 & 1 & 0 & -3
\end{array}
\rightB
\end{equation*}

\end{itemize}