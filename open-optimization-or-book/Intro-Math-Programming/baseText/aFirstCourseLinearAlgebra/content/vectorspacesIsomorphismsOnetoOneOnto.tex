\subsection{One to One and Onto Transformations}

Recall the following definitions, given here in terms of vector spaces.

\begin{definition}{One to One}{onetoonevectorspace}
Let $V, W$ be vector spaces with $\vect{v}_1, \vect{v}_2$ vectors in $V$. Then a linear transformation $T: V \mapsto W$ is called one to one if whenever $\vect{v}_1 \neq \vect{v}_2$ it follows that 
\[
T(\vect{v}_1) \neq T (\vect{v}_2)
\]
\end{definition}

\begin{definition}{Onto}{ontovectorspace}
Let $V, W$ be vector spaces. Then a linear transformation $T: V \mapsto W$ is called onto if for all $\vect{w} \in \vect{W}$ there exists $\vect{v} \in V$ such that $T(\vect{v}) = \vect{w}$. 
\end{definition}

Recall that every linear transformation $T$ has the property that $T(\vect{0})=\vect{0}$. This will be necessary to prove the following useful lemma. 

\begin{lemma}{One to One}{onetooneabstract}
The assertion that a linear transformation $T$ is one to one is equivalent to
saying that if $T(\vect{v})=\vect{0},$ then $\vect{v}=0.$ 
\end{lemma}

\begin{proof}
Suppose first that $T$ is one to one. 
\begin{equation*}
T(\vect{0})=T\left( \vect{0}+\vect{0}\right) =T(\vect{0})+T(\vect{0})
\end{equation*}
and so, adding the additive inverse of $T(\vect{0})$ to both sides, one sees
that $T(\vect{0})=\vect{0}$. Therefore, if $T(\vect{v})=\vect{0},$ it must be the
case that $\vect{v}=\vect{0}$ because it was just shown that $T(\vect{0})=\vect{0}$.

Now suppose that if $T(\vect{v})=\vect{0},$ then $\vect{v}=0.$ If $T(\vect{v})=T(\vect{u}),$ then $T(\vect{v})-T(\vect{u})=T\left( \vect{v}-\vect{u}\right) =\vect{0}$ which
shows that $\vect{v}-\vect{u}=0$ or in other words, $\vect{v}=\vect{u}$. 
\end{proof}

Consider the following example.

\begin{example}{One to One Transformation}{onetoonegeneral}
Let $S:\mathbb{P}_2\to\mathbb{M}_{22}$ be a linear transformation
defined by
\[ S(ax^2+bx+c)
=
\leftB\begin{array}{cc}
a+b & a+c \\ b-c & b+c \end{array}\rightB
\mbox{ for all }
 ax^2+bx+c\in \mathbb{P}_2.\]
Prove that $S$ is one to one but not onto.
\end{example}

\begin{solution}
By definition, 
\[ \ker(S)=\{ax^2+bx+c\in \mathbb{P}_2 ~|~ a+b=0,
a+c=0, b-c=0, b+c=0\}.\]

Suppose $p(x)=ax^2+bx+c\in\ker(S)$.
This leads to a homogeneous system of four equations in three 
variables.  
Putting the augmented matrix in \rref:

\[ \leftB\begin{array}{rrr|c}
1 & 1 & 0 & 0  \\
1 & 0 & 1 & 0  \\
0 & 1 & -1 & 0  \\
0 & 1 & 1 & 0  \end{array}\rightB
\rightarrow \cdots \rightarrow
\leftB\begin{array}{ccc|c}
1 & 0 & 0 & 0  \\
0 & 1 & 0 & 0  \\
0 & 0 & 1 & 0  \\
0 & 0 & 0 & 0  \end{array}\rightB. \]

The solution is $a=b=c=0$. This tells us that if $S(p(x)) = 0$, then $p(x) = ax^2+bx+c = 0x^2 + 0x + 0 = 0$. Therefore it is one to one. 

To show that $S$ is \textbf{not} onto, find a matrix $A\in\mathbb{M}_{22}$
such that for every $p(x)\in \mathbb{P}_2$, 
$S(p(x))\neq A$.
Let 
\[ A=\leftB\begin{array}{cc} 
0 & 1 \\ 0 & 2 \end{array}\rightB,\]
and suppose $p(x)=ax^2+bx+c\in \mathbb{P}_2$ is such that
$S(p(x))=A$.
Then
\[ \begin{array}{ll}
a+b=0 & a+c=1 \\ b-c=0 & b+c=2 \end{array}\]
Solving this system
\[ \leftB\begin{array}{ccc|c}
1 & 1 & 0 & 0  \\
1 & 0 & 1 & 1  \\
0 & 1 & -1 & 0  \\
0 & 1 & 1 & 2  \end{array}\rightB
\rightarrow 
\leftB\begin{array}{rrr|r}
1 & 1 & 0 & 0  \\
0 & -1 & 1 & 1  \\
0 & 1 & -1 & 0  \\
0 & 1 & 1 & 2  \end{array}\rightB. \]

Since the system is inconsistent, there is no $p(x)\in \mathbb{P}_2$ so
that $S(p(x))=A$, and therefore $S$ is not onto.
\end{solution}

\begin{example}{An Onto Transformation}{onto}
Let $T:\mathbb{M}_{22}\to\mathbb{R}^2$ be a linear transformation defined by
\[ T\leftB\begin{array}{cc}
a & b \\ c & d \end{array}\rightB
=
\leftB\begin{array}{c}
a+d \\ b+c \end{array}\rightB
\mbox{ for all }
\leftB\begin{array}{cc}
a & b \\ c & d \end{array}\rightB \in\mathbb{M}_{22}.\]
Prove that $T$ is onto but not one to one. 
\end{example}

\begin{solution}
Let $\leftB\begin{array}{c} x \\ y \end{array}\rightB$ be an arbitrary vector in $\mathbb{R}^2$. 
Since 
$T\leftB\begin{array}{cc} x & y \\ 0 & 0 \end{array}\rightB
=\leftB\begin{array}{c} x \\ y \end{array}\rightB$,
$T$  is onto.

By Lemma \ref{lem:onetooneabstract} $T$ is one to one if and only if $T(A) = \vect{0} $ implies that $A = 0$ the zero matrix.
Observe that
\[
T \left( \leftB\begin{array}{cc} 1 & 0 \\ 0 & -1 \end{array}\rightB \right)
=
\leftB \begin{array}{c}
1 + -1 \\
0 + 0 
\end{array}
\rightB
=
\leftB \begin{array}{c}
0 \\
0 
\end{array}
\rightB
\]
There exists a nonzero matrix $A$ such that $T(A) = \vect{0}$. It follows that $T$ is not one to one.
\end{solution}

The following example demonstrates that a one to one transformation preserves linear independence.

\begin{example}{One to One and Independence}{preservesindependence}
Let $V$ and $W$ be vector spaces and $T: V \mapsto W$ a linear
transformation.
Prove that if $T$ is one to one and
$\{\vect{v}_1, \vect{v}_2, \ldots, \vect{v}_k\}$ is an independent
subset of $V$, then
$\{T(\vect{v}_1), T(\vect{v}_2), \ldots, T(\vect{v}_k)\}$ is an independent
subset of $W$.
\end{example}

\begin{solution}
Let $\vect{0}_V$ and $\vect{0}_W$ denote the zero vectors of $V$ and $W$,
respectively.
Suppose that 

\[ a_1T(\vect{v}_1) + a_2T(\vect{v}_2) +\cdots +a_kT(\vect{v}_k) =\vect{0}_W \]

for some $a_1, a_2, \ldots, a_k\in\mathbb{R}$.
Since linear transformations preserve linear combinations (addition
and scalar multiplication),

\[ T(a_1\vect{v}_1 + a_2\vect{v}_2 +\cdots +a_k\vect{v}_k) =\vect{0}_W. \]

Now, since $T$ is one to one, $\ker(T)=\{\vect{0}_V\}$, and thus

\[ a_1\vect{v}_1 + a_2\vect{v}_2 +\cdots +a_k\vect{v}_k =\vect{0}_V. \]

\noindent However, $\{\vect{v}_1, \vect{v}_2, \ldots, \vect{v}_k\}$ is independent so $a_1=a_2=\cdots=a_k=0$.
Therefore, $\{T(\vect{v}_1), T(\vect{v}_2), \ldots, T(\vect{v}_k)\}$
is independent.
\end{solution}

A similar claim can be made regarding onto transformations. In this case, an onto transformation preserves a spanning set.

\begin{example}{Onto and Spanning}{preservesspanning}
Let $V$ and $W$ be vector spaces and $T:V\to W$ a linear
transformation.
Prove that if $T$ is onto and
$V=\func{span}\{\vect{v}_1, \vect{v}_2, \ldots, \vect{v}_k\}$,
then
\[ W=\func{span}\{T(\vect{v}_1), T(\vect{v}_2), \ldots, T(\vect{v}_k)\}.\]
\end{example}

\begin{solution}
Suppose that $T$ is onto and let $\vect{w}\in W$.  
Then there exists $\vect{v}\in V$ such that $T(\vect{v})=\vect{w}$.
Since $V=\func{span}\{\vect{v}_1, \vect{v}_2, \ldots, \vect{v}_k\}$, there
exist $a_1, a_2, \ldots a_k\in\mathbb{R}$ such that 
$\vect{v} = a_1\vect{v}_1 + a_2\vect{v}_2 + \cdots + a_k\vect{v}_k$.
Using the fact that $T$ is a linear transformation,

\begin{eqnarray*}
\vect{w}=T(\vect{v})
& = & T(a_1\vect{v}_1 + a_2\vect{v}_2 + \cdots + a_k\vect{v}_k) \\
& = & a_1T(\vect{v}_1) + a_2T(\vect{v}_2) + \cdots + a_kT(\vect{v}_k),
\end{eqnarray*}

i.e., $\vect{w}\in\func{span}\{T(\vect{v}_1), T(\vect{v}_2), \ldots, T(\vect{v}_k)\}$,
and thus 

\[ W\subseteq \func{span}\{T(\vect{v}_1), T(\vect{v}_2), \ldots, T(\vect{v}_k)\}.\]

Since $T(\vect{v}_1), T(\vect{v}_2), \ldots, T(\vect{v}_k)\in W$, 
it follows from
that
$\func{span}\{T(\vect{v}_1), T(\vect{v}_2), \ldots, T(\vect{v}_k)\}\subseteq W$,
and therefore
$W=\func{span}\{T(\vect{v}_1), T(\vect{v}_2), \ldots, T(\vect{v}_k)\}$.
\end{solution}
