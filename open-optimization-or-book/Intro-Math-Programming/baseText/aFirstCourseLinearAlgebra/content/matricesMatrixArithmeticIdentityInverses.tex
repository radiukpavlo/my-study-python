\subsection{The Identity and Inverses}

There is a special matrix, denoted $I$, which is called
\index{matrix!identity} to as the \textbf{identity matrix}. The identity matrix is always a square
matrix, and it has
the property that there are ones down the main diagonal and zeroes
elsewhere. Here are some identity matrices of various sizes.
\begin{equation*}
\leftB 1\rightB ,\leftB
\begin{array}{cc}
1 & 0 \\
0 & 1
\end{array}
\rightB ,\leftB
\begin{array}{ccc}
1 & 0 & 0 \\
0 & 1 & 0 \\
0 & 0 & 1
\end{array}
\rightB ,\leftB
\begin{array}{cccc}
1 & 0 & 0 & 0 \\
0 & 1 & 0 & 0 \\
0 & 0 & 1 & 0 \\
0 & 0 & 0 & 1
\end{array}
\rightB 
\end{equation*}
The first is the $1\times 1$ identity matrix, the second is the $2\times 2$
identity matrix, and so on. By extension, you can likely see
what the $n\times n$ identity matrix would be. When it is necessary to distinguish 
which size of identity matrix is being discussed, we will use the 
notation $I_n$ for the $n \times n$ identity matrix. 

The identity matrix is so important that there
is a special symbol to denote the $ij^{th}$ entry of the identity matrix. This symbol is given by 
$I_{ij}=\delta _{ij}$ where $\delta _{ij}$ is the \textbf{Kronecker symbol}
defined
\index{Kronecker symbol} by
\begin{equation*}
\delta _{ij}=\left\{
\begin{array}{c}
1
\text{ if }i=j \\
0\text{ if }i\neq j
\end{array}
\right.
\end{equation*}

$I_n$ is called the \textbf{identity matrix} because it is a \textbf{multiplicative identity} in the following sense.

\begin{lemma}{Multiplication by the Identity Matrix}{multbyidentity}
Suppose $A$ is an $m\times n$ matrix and $I_{n}$ is the $n\times n$ identity
matrix\textbf{.} Then $AI_{n}=A.$ If $I_{m}$ is the $m\times m$ identity
matrix, it also follows that $I_{m}A=A.$
\end{lemma}

\ifdefined\showproofs
\begin{proof}
The $(i,j)$-entry of $AI_n$ is given by:
\begin{equation*}
\sum_{k}a_{ik}\delta _{kj}=a_{ij}
\end{equation*}
and so $AI_{n}=A.$ The other case is left as an exercise for you. 
\end{proof}
\fi
We now define the matrix operation which in some ways plays the role of division. 

\begin{definition}{The Inverse of a Matrix}{invertiblematrix}
A square  $n\times n$ matrix $A$ is said to have an \textbf{inverse} $A^{-1}$
if and only if 
\begin{equation*}
AA^{-1}=A^{-1}A=I_n
\end{equation*}
In this case, the matrix $A$ is called
\index{matrix!inverse} \textbf{invertible}.
\index{matrix!invertible}
\end{definition}

Such a  matrix $A^{-1}$ will have the same size as the matrix $A$. 
It is very important to observe that the inverse of a matrix, if it exists,
is unique. Another way to think of this is that if it acts like the inverse,
then it $\textbf{is}$ the inverse.

\begin{theorem}{Uniqueness of Inverse}{uniqueinverse}
Suppose $A$ is an $n \times\ n$ matrix such that an inverse  $A^{-1}$ exists. Then there is only one such 
inverse matrix. 
That is, given any matrix $B$ such that $AB=BA=I$, $B=A^{-1}$.
\end{theorem}

\ifdefined\showproofs
\begin{proof} In this proof, it is assumed that $I$ is the $n \times n$ identity matrix. 
Let $A, B$ be $n \times n$ matrices such that $A^{-1}$ exists and $AB=BA=I$. 
We want to show that $A^{-1} = B$. 
Now using properties we have seen, we get: 
\begin{equation*}
A^{-1}=A^{-1}I=A^{-1}\left( AB\right) =\left( A^{-1}A\right) B=IB=B
\end{equation*}
Hence, $A^{-1} = B$ which tells us that the inverse is unique.
\end{proof}
\fi
The next example demonstrates how to check the inverse of a matrix. 

\begin{example}{Verifying the Inverse of a Matrix}{verifyinginverse}
Let $A=\leftB
\begin{array}{rr}
1 & 1 \\
1 & 2
\end{array}
\rightB .$ Show $\leftB
\begin{array}{rr}
2 & -1 \\
-1 & 1
\end{array}
\rightB $ is the inverse of $A.$
\end{example}

\begin{solution} To check this, multiply
\begin{equation*}
\leftB
\begin{array}{rr}
1 & 1 \\
1 & 2
\end{array}
\rightB \leftB
\begin{array}{rr}
2 & -1 \\
-1 & 1
\end{array}
\rightB =\allowbreak \leftB
\begin{array}{rr}
1 & 0 \\
0 & 1
\end{array}
\rightB = I 
\end{equation*}
and
\begin{equation*}
\leftB
\begin{array}{rr}
2 & -1 \\
-1 & 1
\end{array}
\rightB \leftB
\begin{array}{rr}
1 & 1 \\
1 & 2
\end{array}
\rightB =\allowbreak \leftB
\begin{array}{rr}
1 & 0 \\
0 & 1
\end{array}
\rightB = I 
\end{equation*}
showing that this matrix is indeed the inverse of $A.$
\end{solution}

Unlike ordinary multiplication of numbers, it can happen that $A\neq 0$ but 
$A$ may fail to have an inverse. This is illustrated in the following example.

\begin{example}{A Nonzero Matrix With No Inverse}{noninvertiblematrix}
Let $A=\leftB
\begin{array}{rr}
1 & 1 \\
1 & 1
\end{array}
\rightB .$ Show that $A$ does not have an inverse.
\end{example}

\begin{solution} One might think $A$ would have an inverse because it does not equal zero.
However, note that 
\begin{equation*}
\leftB
\begin{array}{rr}
1 & 1 \\
1 & 1
\end{array}
\rightB \leftB
\begin{array}{r}
-1 \\
1
\end{array}
\rightB =\leftB
\begin{array}{r}
0 \\
0
\end{array}
\rightB
\end{equation*}
If $A^{-1}$ existed, we would have the following
\begin{eqnarray*}
\leftB
\begin{array}{r}
0 \\
0
\end{array}
\rightB &=& A^{-1}\left( \leftB
\begin{array}{r}
0 \\
0
\end{array}
\rightB \right) \\
&=& A^{-1}\left( A\leftB
\begin{array}{r}
-1 \\
1
\end{array}
\rightB \right) \\
&=&\left( A^{-1}A\right) \leftB
\begin{array}{r}
-1 \\
1
\end{array}
\rightB \\
&=&I\leftB
\begin{array}{r}
-1 \\
1
\end{array}
\rightB \\
&=&\leftB
\begin{array}{r}
-1 \\
1
\end{array}
\rightB
\end{eqnarray*}
This says that 
\begin{equation*}
\leftB
\begin{array}{r}
0 \\
0
\end{array}
\rightB
=
\leftB
\begin{array}{r}
-1 \\
1
\end{array}
\rightB
\end{equation*}
which is impossible! Therefore, $A$ does not have an inverse. 
\end{solution}

In the next section, we will explore how to find the inverse of a matrix, if it exists. 