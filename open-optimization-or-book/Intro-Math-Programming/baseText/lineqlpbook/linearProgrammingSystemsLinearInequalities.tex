%By Kevin Cheung
%The book is licensed under the
%\href{http://creativecommons.org/licenses/by-sa/4.0/}{Creative Commons
%Attribution-ShareAlike 4.0 International License}.
%
%This file has been modified by Robert Hildebrand 2020.  
%CC BY SA 4.0 licence still applies.

\section{Solving systems of linear
inequalities}\label{solving-systems-of-linear-inequalities}

Before we can solve a linear programming problem, we should be able to
solve the seemingly simpler problem of finding a feasible solution. We
will now consider how one can determine if a system of linear
inequalities has a solution.

For the sake of illustrating the principles involved, we limit ourselves
to systems consisting of only \(\geq\)-inequalities. Extending the
method to work with any systems of linear constraints is left as an
exercise.

Suppose that we want to determine if there exist \(x,y\in\mathbb{R}\)
satisfying
\begin{align*}
x + y & \geq 0 \\
2x + y & \geq 2 \\
-x + y  & \geq 1 \\
-x + 2y & \geq -1.
\end{align*}

The key is to take one of the variables and see how it is constrained by
the remaining variables. We ``isolate'' \(x\) by rewriting the system to
the equivalent system
\begin{align*}
x & \geq -y  \\
x & \geq 1 - \frac{1}{2}y \\
x & \leq -1 +y \\
x & \leq 1 + 2y.
\end{align*}

Hence, \(x\) is constrained by the lower bounds \(-y\) and
\(1 - \frac{1}{2}y\) and the upper bounds \(-1 +y\) and \(1 + 2y\).
Therefore, we can find a value for \(x\) satisfying these bounds if and
only if each of the upper bounds is at least each of the lower bounds;
that is,
\begin{align*}
-1 + y & \geq -y \\
-1 + y & \geq 1 - \frac{1}{2}y \\
1 + 2y & \geq -y \\
1 + 2y & \geq 1 - \frac{1}{2}y.
\end{align*}

Simplifying this system gives
\begin{align*}
2y & \geq 1 \\
\frac{3}{2}y & \geq 2  \\
3y & \geq -1 \\
\frac{5}{2}y & \geq 0,
\end{align*}

or more simply,
\begin{align*}
y & \geq \frac{1}{2} \\
y & \geq \frac{4}{3}  \\
y & \geq -\frac{1}{3} \\
y & \geq 0.
\end{align*}

Note that this system does not contain the variable \(x\) and it has a
solution if and only if \(y \geq \frac{4}{3}\). Hence, the original
system has a solution if and only if \(y \geq \frac{4}{3}\). If we set
\(y = 2\), for example, then \(x\) must satisfy
\begin{align*}
x & \geq - 2  \\
x & \geq 0 \\
x & \leq 1 \\
x & \leq 5.
\end{align*}

Thus, we can pick \(x\) to be any value in the closed interval
\([0,1]\). In particular,
\(\begin{bmatrix} x\\ y\end{bmatrix} = \begin{bmatrix} 0 \\ 2\end{bmatrix}\)
is \emph{one} solution to the given system of linear inequalities. There
could be other solutions.

The above example illustrates the process of solving a system of linear
inequaltiies by constructing a system that has a reduced number of
variables. As the number of variables is finite, the process can be
repeated until we obtain a system whose solvability is apparent (as in
the one-variable case).

Observe that the pairing of an upper bound constraint of the form
\(x \leq q\) and a lower bound constraint of the form \(x \geq p\) to
obtain \(q \geq p\) is equivalent to adding the inequalities
\(-x \geq -q\) and \(x \geq p\). This observation leads to the
following:

\hypertarget{fm}{\subsection{Fourier-Motzkin Elimination}\label{fm}}

Given: A system of linear inequalities
\[ \sum_{j=1}^n a_{ij} x_j \geq b_i,~~~i = 1,\ldots,m\] where
\(a_{ij}, b_i \in \mathbb{R}\) for \(i = 1,\ldots,m\) and
\(j = 1,\ldots,n\),

Eliminate \(x_k\) for some \(k \in \{1,\ldots,n\}\) using the following
steps:

\begin{enumerate}
\def\labelenumi{\arabic{enumi}.}
\item
  For each \(j \in \{1,\ldots m\}\),

  \begin{itemize}
  \tightlist
  \item
    if \(a_{jk} \gt 0\), multiply the \(j\)th inequality by
    \(\frac{1}{a_{jk}}\),
  \item
    if \(a_{jk} \lt 0\), multiply the \(j\)th inequality by
    \(-\frac{1}{a_{jk}}\)
  \end{itemize}
\item
  Form a new system of inequalities as follows:

  \begin{itemize}
  \tightlist
  \item
    copy down all the inequalities in which the coefficient of \(x_k\)
    is 0
  \item
    for each inequality in which \(x_k\) has positive coefficent and for
    each inequality in which \(x_k\) has negative coefficient, obtain a
    new inequality by adding them together.
  \end{itemize}
\end{enumerate}

\textbf{Remarks.}

\begin{enumerate}
\def\labelenumi{\arabic{enumi}.}
\item
  Step 1 is to ensure that all the nonzero coefficients of \(x_k\) are
  \(1\) or \(-1\).
\item
  The new system formed in Step 2 will not contain the variable \(x_k\).
  Furthermore, if \(x_1^*,\ldots, x_n^*\) is a solution to the original
  system, then \(x_1^*,\ldots,x_{k-1}^*, x_{k+1}^*,\ldots, x_n^*\) is a
  solution to the new system. And if
  \(x_1^*,\ldots,x_{k-1}^*, x_{k+1}^*,\ldots, x_n^*\) is a solution to
  the new system, then there exists \(x_k^*\) such that
  \(x_1^*,\ldots, x_n^*\) is a solution to the original system. (Why?)
  Hence, the original system has a solution if and only if the new
  system does.
\end{enumerate}

Now, if we apply Fourier-Motzkin elimination repeatedly, we obtain a
system with at most one variable such that it has a solution if and only
if the original system does. Since solving systems of linear
inequalities with at most one variable is easy, we can conclude whether
or not the original system has a solution.

Note that if the coefficients are all rational, the system obtained
after eliminating one variable using Fourier-Motzkin elimination will
also have only rational coefficients.

\begin{example}{}{}
\protect\hypertarget{ex:unnamed-chunk-1}{}{\label{ex:unnamed-chunk-1}}Determine
if the following system of inequalities has a solution:
\begin{align*}
x_1 + x_2 - 2x_3 & \geq 2 ~~~~~~(1)\\
-x_1 - 3x_2 + x_3 & \geq 0~~~~~~(2) \\
x_2 + x_3 & \geq 1~~~~~~(3) 
\end{align*}

We first eliminate \(x_1\). The new system is

\(\begin{array}{rrl} (1) + (2): & -2x_2 - x_3 \geq 2 & ~~~(4) \\  & x_2 + x_3 \geq 1 & ~~~(3) \end{array}\)

We then eliminate \(x_2\). We first normalize the coefficients of
\(x_2\):

\(\begin{array}{rrl} \frac{1}{2}\times (4) & -x_2 - \frac{1}{2}x_3 \geq 1 & ~~~(5) \\  & x_2 + x_3 \geq 1 & ~~~(3) \end{array} \)

So the new system is:

\(\begin{array}{rl} (5)+(3): & \frac{1}{2} x_3 \geq 2 \end{array} \)

So there is a solution. In particular, we can set \(x_3 = 4\). Then we
must have \(x_2 = -3\) and \(x_1 = 13\).
\end{example}

\textbf{Remark.} Note that setting \(x_3\) to another value larger than
\(4\) will lead to different solutions to the system. Since there are
infinitely many different values that we can set \(x_3\) to, there are
infinitely many solutions.

\section*{Exercises}\label{exercises-3}
\addcontentsline{toc}{section}{Exercises}

\begin{enumerate}
\def\labelenumi{\arabic{enumi}.}
\tightlist
\item
  Use Fourier-Motzkin elimination to determine if there exist
  \(x,y,z\in\mathbb{R}\) satisfying

  \begin{eqnarray*}
  x + y + 2z& \geq & 1 \\
  -x + y + z & \geq & 2 \\
  x-y + z  & \geq & 1 \\
  -y - 3z & \geq & 0.
  \end{eqnarray*}
\item
  Let \(\mathbf{a}^1,\ldots, \mathbf{a}^m \in \mathbb{R}^n\). Let
  \(\beta_1,\ldots, \beta_m \in\mathbb{R}\). Let
  \(\lambda_1,\ldots, \lambda_m \geq 0\). Then the inequality
  \(\displaystyle \left(\sum_{i = 1}^{m} \lambda_i {\mathbf{a}^{i}}\right)^\mathsf{T}\mathbf{x} \geq \sum_{i=1}^{m} \lambda_i \beta_i\)
  is called a nonnegative linear combination of the inequalities
  \({\mathbf{a}^{i}}^\mathsf{T} \mathbf{x} \geq \beta_i\),
  \(i = 1,\ldots, m\). Show that any new inequality created by
  Fourier-Motzkin Elimination is a nonnegative linear combination of the
  original inequalities.
\end{enumerate}

\section*{Solutions}\label{solutions-3}
\addcontentsline{toc}{section}{Solutions}

\begin{enumerate}
\def\labelenumi{\arabic{enumi}.}
\item
  We use Fourier-Motzkin elimination to eliminate \(x\). We first copy
  down the inequality \(-y-3z\geq 0\) and then form one new inequality
  by adding the first two inequalities and another by adding the second
  and third inequalities. The resulting system is

  \begin{eqnarray*}
  -y - 3z & \geq & 0 \\
   2y + 3z& \geq & 3 \\
  2z & \geq & 3.
  \end{eqnarray*}

  Note that this system has a solution if and only if the original
  system does.

  We now use Fourier-Motzkin elimination to eliminate \(y\). First we
  multiply the second inequality by \(\frac{1}{2}\) to obtain

  \begin{eqnarray*}
  -y - 3z & \geq & 0 \\
   y + \frac{3}{2}z& \geq & \frac{3}{2} \\
  2z & \geq & 3.
  \end{eqnarray*}

  Eliminating \(y\) gives

  \begin{eqnarray*}
  2z & \geq & 3 \\
   - \frac{3}{2}z& \geq & \frac{3}{2},
  \end{eqnarray*}

  or equivalently,

  \begin{eqnarray*}
  z & \geq & \frac{3}{2} \\
   z& \leq & -1,
  \end{eqnarray*}

  which clearly has no solution. Hence, there is no \(x,y,z\) satisfying
  the original system.
\item
  First of all, observe that a nonnegative linear combination of
  \(\geq\)-inequalites that are themselves nonnegative linear
  combination of the inequalites in
  \(\mathbf{A}\mathbf{x}\geq \mathbf{b}\) is again a nonnegative linear
  combination of inequalities in
  \(\mathbf{A}\mathbf{x}\geq \mathbf{b}\).

  It is easy to see that in Step 1 of Fourier-Motzkin Elimination all
  inequalities are nonnegative linear combinations of the original
  inequalities. For instance, multiplying
  \({\mathbf{a}^{i'}}^\mathsf{T}\mathbf{x} \geq \beta_{i'}\) by
  \(\alpha \gt 0\) is the same as taking the nonnegative linear
  combination
  \(\displaystyle \left(\sum_{i = 1}^m \lambda_i {\mathbf{a}^{i}}\right)^\mathsf{T}\mathbf{x} \geq \sum_{i=1}^m \lambda_i \beta_i\)
  with \(\lambda_i = 0\) for all \(i \neq i'\) and
  \(\lambda_{i'} = \alpha\).

  In Step 2, new inequalities are formed by adding two inequalities from
  Step 1. Hence, they are nonnegative linear combinations of the
  inequalities from Step 1. By the observation at the beginning, they
  are nonnegative linear combinations of the original system.

  \textbf{Remark.} By the observation at the beginning and this result,
  we see that after repeated applications of Fourier-Motzkin
  Elimination, all resulting inequalities are nonnegative linear
  combinations of the original inequalities. This is an important fact
  that will be exploited later.
\end{enumerate}
