%By Kevin Cheung
%The book is licensed under the
%\href{http://creativecommons.org/licenses/by-sa/4.0/}{Creative Commons
%Attribution-ShareAlike 4.0 International License}.
%
%This file has been modified by Robert Hildebrand 2020.  
%CC BY SA 4.0 licence still applies.

\chapter{Solving linear programming problems}\label{fund-lp}

\protect\hyperlink{fm}{Fourier-Motzkin elimination} can actually be used
to solve a linear programming problem though the method is not efficient
and is almost never used in practice. We illustrate the process with an
example.

Consider the following linear programming problem:

\begin{equation}
\begin{array}{rl}
\min & x + y \\
\text{s.t.}
& x + 2y  \geq 2 \\
& 3x + 2y  \geq 6.
\end{array}\label{eq:LP}
\end{equation}

Observe that \eqref{eq:LP} is equivalent to

\begin{equation}
\begin{array}{rl}
\min & z \\
\text{s.t.}
& z - x - y = 0 \\
& x + 2y  \geq 2 \\
& 3x + 2y  \geq 6.
\end{array}\label{eq:LPprime}
\end{equation}

Note that the objective function is replaced with \(z\) and \(z\) is set
to the original objective function in the first constraint of
\eqref{eq:LPprime} since \(z = x+ y\) if and only if \(z-x-y=0\). Then,
solving \eqref{eq:LPprime} is equivalent to finding among all the
solutions to the following system a solution that minimizes \(z\), if it
exists. \[
\begin{array}{rl}
 z - x - y \geq 0 & ~~~(1) \\
-z + x + y \geq 0 & ~~~(2) \\
 x + 2y  \geq 2 &~~~(3)\\
 3x + 2y  \geq 6 & ~~~(4)
\end{array}
\] Since we are interested in the minimum possible value for \(z\) we
use Fourier-Motzking elimination to eliminate the variables \(x\) and
\(y\).

To eliminate \(x\), we first multiply \((4)\) by \(\frac{1}{3}\) to
obtain: \[
\begin{array}{rl}
 z - x - y \geq 0 & ~~~(1) \\
-z + x + y \geq 0 & ~~~(2) \\
 x + 2y  \geq 2 &~~~(3)\\
 x + \frac{2}{3}y  \geq 2 & ~~~(5)
\end{array}
\] Then eliminate \(x\) to obtain \[
\begin{array}{rrl}
(1) + (2):  & 0 \geq 0 \\
(1) + (3):  & z + y \geq 2 & ~~~(6) \\
(1) + (5):  & z - \frac{1}{3} y \geq 2 & ~~~(7) \\
\end{array}
\] Note that there is no need to keep the first inequality. To eliminate
\(y\), we first multiply \((7)\) by \(3\) to obtain: \[
\begin{array}{rl}
  z + y \geq 2 & ~~~(6) \\
  3z - y \geq 6 & ~~~(8) \\
\end{array}
\] Then eliminate \(y\) to obtain \[
\begin{array}{rl}
  4z \geq 8 & ~~~(9) \\
\end{array}
\] Multiplying \((9)\) by \(\frac{1}{4}\) gives \(z \geq 2\). Hence, the
minimum possible value for \(z\) among all the solutions to the system
is \(2\). So the optimal value of \eqref{eq:LPprime} is \(2\). To obtain
an optimal solution, set \(z = 2\). Then we have no choice but to set
\(y = 0\) and \(x = 2\). One can check that \((x,y) = (2,0)\) is a
feasible solution with objective function value \(2\).

We can obtain an independent proof that the optimal value is indeed
\(2\) if we trace back the computations. Note that the inequality
\(z \geq 2\) is given by

\begin{eqnarray*}
\frac{1}{4} (9) 
& \Leftarrow & \frac{1}{4} (6) + \frac{1}{4} (8) \\
& \Leftarrow & \frac{1}{4} (1)+\frac{1}{4}(3) + \frac{3}{4}(7) \\
& \Leftarrow & \frac{1}{4} (1)+\frac{1}{4}(3) + \frac{3}{4}(1)+\frac{3}{4}(5) \\
& \Leftarrow & (1)+ \frac{1}{4}(3) + \frac{1}{4} (4)  \\
\end{eqnarray*}

This shows that \(\frac{1}{4}(3) + \frac{1}{4} (4)\) gives the
inequality \(x+y \geq 2\). Hence, no feasible solution to \eqref{eq:LP}
can have objective function value less than \(2\). But we have found one
feasible solution with objective function value \(2\). Hence, \(2\) is
the optimal value.



