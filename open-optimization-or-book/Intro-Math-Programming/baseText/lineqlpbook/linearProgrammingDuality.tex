%By Kevin Cheung
%The book is licensed under the
%\href{http://creativecommons.org/licenses/by-sa/4.0/}{Creative Commons
%Attribution-ShareAlike 4.0 International License}.
%
%This file has been modified by Robert Hildebrand 2020.  
%CC BY SA 4.0 licence still applies.

\section{Linear programming duality}\label{linear-programming-duality}

Consider the following problem:

\begin{equation}
\begin{array}{rl}
\min & \vec{c}^\T\vec{x} \\
\mbox{s.t.} & \mm{A}\vec{x} \geq \vec{b}.
\label{eq:duality-primal}
\end{array}
\end{equation}

In the remark at the end of Chapter \ref{fund-lp}, we saw that if
\eqref{eq:duality-primal} has an optimal solution, then there exists
\(\vec{y}^*\in\R^m\) such that \(\vec{y}^* \geq 0\),
\({\vec{y}^*}^\T\mm{A} = \vec{c}^\T\), and
\({\vec{y}^*}^\T\vec{b} = \gamma\) where \(\gamma\) denotes the optimal
value of \eqref{eq:duality-primal}.

Take any \(\vec{y}\in\R^m\) satisfying \(\vec{y} \geq \vec{0}\) and
\(\vec{y}^\T\mm{A} = \vec{c}^\T\). Then we can infer from
\(\mm{A}\vec{x}\geq \vec{b}\) the inequality
\(\vec{y}^\T\mm{A}\vec{x} \geq \vec{y}^\T \vec{b}\), or more simply,
\(\vec{c}^\T\vec{x} \geq \vec{y}^\T \vec{b}\). Thus, for any such
\(\vec{y}\), \(\vec{y}^\T \vec{b}\) gives a lower bound for the
objective function value of any feasible solution to
\eqref{eq:duality-primal}. Since \(\gamma\) is the optimal value of
\((P)\), we must have \(\gamma \geq \vec{y}^\T\vec{b}\).

As \({\vec{y}^*}^\T\vec{b} = \gamma\), we see that \(\gamma\) is the
optimal value of

\begin{equation}
\begin{array}{rl}
\max & \vec{y}^\T\vec{b} \\
\mbox{s.t.} & \vec{y}^\T\mm{A} = \vec{c}^\T \\
&\vec{y} \geq \vec{0}.
\end{array}\label{eq:duality-dual}
\end{equation}

Note that \eqref{eq:duality-dual} is a linear programming problem! We call
it the \textbf{dual problem} of the \textbf{primal problem}
\eqref{eq:duality-primal}. We say that the dual variable \(y_i\) is
\textbf{associated} with the constraint
\({\vec{a}^{(i)}}^\T \vec{x} \geq b_i\) where \({\vec{a}^{(i)}}^\T\)
denotes the \(i\)th row of \(\mm{A}\).

In other words, we define the dual problem of \eqref{eq:duality-primal} to
be the linear programming problem \eqref{eq:duality-dual}. In the
discussion above, we saw that if the primal problem has an optimal
solution, then so does the dual problem and the optimal values of the
two problems are equal. Thus, we have the following result:

\begin{theorem}{strong-duality-special}{}
\protect\hypertarget{thm:strong-duality-special}{}{\label{thm:strong-duality-special}}
Suppose that \eqref{eq:duality-primal} has an optimal solution. Then
\eqref{eq:duality-dual} also has an optimal solution and the optimal
values of the two problems are equal.
\end{theorem}

At first glance, requiring all the constraints to be
\(\geq\)-inequalities as in \eqref{eq:duality-primal} before forming the
dual problem seems a bit restrictive. We now see how the dual problem of
a primal problem in general form can be defined. We first make two
observations that motivate the definition.

\textbf{Observation 1}

Suppose that our primal problem contains a mixture of all types of
linear constraints:

\begin{equation}
\begin{array}{rl}
\min & \vec{c}^\T\vec{x}  \\
\mbox{s.t.} & \mm{A}\vec{x}\geq \vec{b} \\
& \mm{A'}\vec{x} \leq \vec{b'} \\
& \mm{A''}\vec{x} = \vec{b''}
\end{array} \label{eq:P-prime}
\end{equation}

where \(\mm{A} \in \R^{m\times n}\), \(\vec{b} \in \R^{m}\),
\(\mm{A}' \in \R^{m'\times n}\), \(\vec{b}' \in \R^{m'}\),
\(\mm{A}'' \in \R^{m''\times n}\), and \(\vec{b}'' \in \R^{m''}\).

We can of course convert this into an equivalent problem in the form of
\eqref{eq:duality-primal} and form its dual.\\
However, if we take the point of view that the function of the dual is
to infer from the constraints of \eqref{eq:P-prime} an inequality of the
form \(\vec{c}^\T \vec{x} \geq \gamma\) with \(\gamma\) as large as
possible by taking an appropriate linear combination of the constraints,
we are effectively looking for \(\vec{y} \in \R^{m}\),
\(\vec{y} \geq \vec{0}\), \(\vec{y}' \in \R^{m'}\),
\(\vec{y}' \leq \vec{0}\), and \(\vec{y}'' \in \R^{m''}\), such that
\[\vec{y}^\T\mm{A}
+\vec{y'}^\T\mm{A'}
+\vec{y''}^\T\mm{A''} = \vec{c}^\T\] with
\(\vec{y}^\T\vec{b} +\vec{y'}^\T\vec{b'} +\vec{y''}^\T\vec{b''}\) to be
maximized.

(The reason why we need \(\vec{y'} \leq \vec{0}\) is because inferring a
\(\geq\)-inequality from \(\mm{A}'\vec{x} \leq \vec{b}'\) requires
nonpositive multipliers. There is no restriction on \(\vec{y''}\)
because the constraints \(\mm{A}''\vec{x} = \vec{b}''\) are equalities.)

This leads to the dual problem:

\begin{equation}
\begin{array}{rl}
\max~ & \vec{y}^\T\vec{b} 
+\vec{y'}^\T\vec{b'}
+\vec{y''}^\T\vec{b''} \\
\mbox{s.t.} ~
& \vec{y}^\T\mm{A}
+\vec{y'}^\T\mm{A'}
+\vec{y''}^\T\mm{A''} = \vec{c}^\T \\
&~~~~\vec{y} \geq \vec{0} \\
&~~~~\vec{y'} \leq \vec{0}.
\end{array} \label{eq:D-prime}
\end{equation}

In fact, we could have derived this dual by applying the definition of
the dual problem to

\begin{equation*}
\begin{array}{rl}
\min ~& \vec{c}^\T\vec{x} \\
\mbox{s.t.}  ~
& \begin{bmatrix} 
 \mm{A} \\
 -\mm{A'} \\
 \mm{A''} \\
 -\mm{A''}
\end{bmatrix} \vec{x}
\geq
\begin{bmatrix}
\vec{b} \\
-\vec{b'} \\
\vec{b''} \\
-\vec{b''}
\end{bmatrix},
\end{array}
\end{equation*}

which is equivalent to \eqref{eq:P-prime}. The details are left as an
exercise.

\textbf{Observation 2}

Consider the primal problem of the following form:

\begin{equation}
\begin{array}{rl}
\min ~& \vec{c}^\T\vec{x} \\
\mbox{s.t.} ~& \mm{A}\vec{x} \geq \vec{b} \\
 & x_i \geq 0 ~~i \in P \\
 & x_i \leq 0 ~~i \in N 
\end{array}\label{eq:P-dbl-prime}
\end{equation}

where \(P\) and \(N\) are disjoint subsets of \(\{1,\ldots,n\}\). In
other words, constraints of the form \(x_i \geq 0\) or \(x_i \leq 0\)
are separated out from the rest of the inequalities.

Forming the dual of \eqref{eq:P-dbl-prime} as defined under Observation 1,
we obtain the dual problem

\begin{equation}
\begin{array}{rll}
\max & \vec{y}^\T\vec{b} \\
\mbox{s.t.} & 
 \vec{y}^\T\vec{a}^{(i)} = c_i & i \in \{1,\ldots,n\}\backslash 
(P\cup N) \\
&\vec{y}^\T\vec{a}^{(i)} + p_i = c_i & i \in P \\
& \vec{y}^\T\vec{a}^{(i)} + q_i = c_i & i \in N \\
& p_i \geq 0 & i \in P \\
& q_i \leq 0 & i \in N \\
\end{array} \label{eq:D-tilde}
\end{equation}

where \(\vec{y} = \begin{bmatrix} y_1\\ \vdots \\ y_m\end{bmatrix}\).
Note that this problem is equivalent to the following without the
variables \(p_i\), \(i \in P\) and \(q_i\), \(i \in N\):

\begin{equation}
\begin{array}{rll}
\max & \vec{y}^\T\vec{b} \\
\mbox{s.t.} & 
 \vec{y}^\T\vec{a}^{(i)} = c_i & i \in \{1,\ldots,n\}\backslash 
(P\cup N) \\
&\vec{y}^\T\vec{a}^{(i)} \leq c_i & i \in P      \\
& \vec{y}^\T\vec{a}^{(i)} \geq c_i & i \in N,     \\
\end{array}
\end{equation}

which can be taken as the dual problem of \eqref{eq:P-dbl-prime} instead
of \eqref{eq:D-tilde}. The advantage here is that it has fewer variables
than \eqref{eq:D-tilde}.

Hence, the dual problem of

\begin{equation*}
\begin{array}{rl}
\min & \vec{c}^\T\vec{x}  \\
\mbox{s.t.} & \mm{A}\vec{x} \geq \vec{b} \\
& \vec{x} \geq \vec{0}
\end{array}
\end{equation*}

is simply

\begin{equation*}
\begin{array}{rl}
\max & \vec{y}^\T\vec{b} \\
\mbox{s.t.} & \vec{y}^\T\mm{A} \leq \vec{c}^\T \\
& \vec{y} \geq \vec{0}.
\end{array}
\end{equation*}

As we can see from bove, there is no need to associate dual variables to
constraints of the form \(x_i \geq 0\) or \(x_i \leq 0\) provided we
have the appropriate types of constraints in the dual problem. Combining
all the observations lead to the definition of the dual problem for a
primal problem in general form as discussed next.

\hypertarget{primal-dual}{\subsection{The dual problem}\label{primal-dual}}

Let \(\mm{A} \in \R^{m\times n}\), \(\vec{b} \in \R^m\),
\(\vec{c} \in \R^n\). Let \({\vec{a}^{(i)}}^\T\) denote the \(i\)th row
of \(\mm{A}\). Let \(\mm{A}_j\) denote the \(j\)th column of \(\mm{A}\).

Let \((P)\) denote the minimization problem with variables in the tuple
\(\vec{x} = \begin{bmatrix} x_1\\ \vdots \\ x_n\end{bmatrix}\) given as
follows:

\begin{itemize}
\item
  The objective function to be minimized is \(\vec{c}^\T \vec{x}\)
\item
  The constraints are

  \begin{equation*}{\vec{a}^{(i)}}^\T\vec{x}~\sqcup_i~b_i
  \end{equation*}

  where \(\sqcup_i\) is \(\leq\), \(\geq\), or \(=\) for
  \(i = 1,\ldots, m\).
\item
  For each \(j \in \{1,\ldots,n\}\), \(x_j\) is constrained to be
  nonnegative, nonpositive, or free (i.e.~not constrained to be
  nonnegative or nonpositive.)
\end{itemize}

Then the \textbf{dual problem} is defined to be the maximization problem
with variables in the tuple
\(\vec{y} = \begin{bmatrix} y_1\\ \vdots\\ y_m\end{bmatrix}\) given as
follows:

\begin{itemize}
\item
  The objective function to be maximized is \(\vec{y}^\T \vec{b}\)
\item
  For \(j = 1,\ldots, n\), the \(j\)th constraint is

  \begin{equation*}
  \left \{\begin{array}{ll}
  \vec{y}^\T\mm{A}_j \leq c_j & \text{if } x_j \text{ is constrained to 
  be nonnegative} \\
  \vec{y}^\T\mm{A}_j \geq c_j & \text{if } x_j \text{ is constrained to 
  be nonpositive} \\
  \vec{y}^\T\mm{A}_j = c_j & \text{if } x_j \text{ is free}.
  \end{array}\right.
  \end{equation*}
\item
  For each \(i \in \{1,\ldots,m\}\), \(y_i\) is constrained to be
  nonnegative if \(\sqcup_i\) is \(\geq\); \(y_i\) is constrained to be
  nonpositive if \(\sqcup_i\) is \(\leq\); \(y_i\) is free if
  \(\sqcup_i\) is \(=\).
\end{itemize}

The following table can help remember the above.

\begin{longtable}[]{@{}cc@{}}
\toprule
Primal (min) & Dual (max)\tabularnewline
\midrule
\endhead
\(\geq\) constraint & \(\geq 0\) variable\tabularnewline
\(\leq\) constraint & \(\leq 0\) variable\tabularnewline
\(=\) constraint & \(\text{free}\) variable\tabularnewline
\(\geq 0\) variable & \(\leq\) constraint\tabularnewline
\(\leq 0\) variable & \(\geq\) constraint\tabularnewline
\(\text{free}\) variable & \(=\) constraint\tabularnewline
\bottomrule
\end{longtable}

Below is an example of a primal-dual pair of problems based on the above
definition:

Consider the primal problem: \[\begin{array}{rrcrcrcl}
\mbox{min} &  x_1 & - & 2x_2 & + & 3x_3 & \\
\mbox{s.t.} & -x_1 &   &      & + & 4x_3 &  =   &5 \\
            & 2x_1 & + & 3x_2 & - & 5x_3 & \geq &  6 \\
            &      &   & 7x_2 &   &      & \leq &  8 \\
            &  x_1 &   &      &   &      & \geq &  0 \\
            &     &    & x_2  &   &      & &      \mbox{free} \\
            &     &    &      &   & x_3  & \leq & 0.\\
\end{array}\]

Here,
\(\mm{A}= \begin{bmatrix}  -1 & 0 & 4 \\  2 & 3 & -5 \\  0 & 7 & 0 \end{bmatrix}\),
\(\vec{b} = \begin{bmatrix}5 \\6\\8\end{bmatrix}\), and
\(\vec{c} = \begin{bmatrix}1 \\-2\\3\end{bmatrix}\).

The primal problem has three constraints. So the dual problem has three
variables. As the first constraint in the primal is an equation, the
corresponding variable in the dual is free. As the second constraint in
the primal is a \(\geq\)-inequality, the corresponding variable in the
dual is nonnegative. As the third constraint in the primal is a
\(\leq\)-inequality, the corresponding variable in the dual is
nonpositive. Now, the primal problem has three variables. So the dual
problem has three constraints. As the first variable in the primal is
nonnegative, the corresponding constraint in the dual is a
\(\leq\)-inequality. As the second variable in the primal is free, the
corresponding constraint in the dual is an equation. As the third
variable in the primal is nonpositive, the corresponding constraint in
the dual is a \(\geq\)-inequality. Hence, the dual problem is:
\[\begin{array}{rrcrcrcl}
\mbox{max} & 5y_1 & + & 6y_2 & + & 8y_3 & \\
\mbox{s.t.} & -y_1 & + & 2y_2 &   &      & \leq &  1 \\
            &      &   & 3y_2 & + & 7y_3 & = & -2 \\
            & 4y_1 & - & 5y_2 &   &      & \geq &  3 \\
            &  y_1 &   &      &   &      &      &  \mbox{free} \\ 
            &     &    & y_2  &   &      & \geq & 0 \\
            &     &    &      &   & y_3  & \leq & 0.\\
\end{array}\]

\textbf{Remarks.} Note that in some books, the primal problem is always
a maximization problem. In that case, what is our primal problem is
their dual problem and what is our dual problem is their primal problem.

One can now prove a more general version of Theorem
\ref{thm:strong-duality-special} as stated below. The details are left
as an exercise.

\begin{theorem}{Duality Theorem for Linear Programming}{}
\protect\hypertarget{thm:strong-duality}{}{\label{thm:strong-duality}
\iffalse (Duality Theorem for Linear Programming) \fi{} } Let (P) and
(D) denote a primal-dual pair of linear programming problems. If either
(P) or (D) has an optimal solution, then so does the other. Furthermore,
the optimal values of the two problems are equal.
\end{theorem}

Theorem \ref{thm:strong-duality} is also known informally as
\textbf{strong duality}.

\subsection*{Exercises}\label{exercises-6}
\addcontentsline{toc}{section}{Exercises}

\begin{enumerate}
\def\labelenumi{\arabic{enumi}.}
\item
  Write down the dual problem of \[\begin{array}{rrcrcl}
  \min & 4x_1 & - & 2x_2 \\
  \text{s.t.} 
   & x_1 & + & 2x_2 & \geq & 3\\
   & 3x_1 & - & 4x_2 & = & 0 \\
   &  & & x_2 & \geq & 0.
  \end{array}\]
\item
  Write down the dual problem of the following: \[
  \begin{array}{rrcrcrcl}
  \min  &    &  &3x_2  & + & x_3 \\
  \mbox{s.t.}
   & x_1 & + & x_2 & + & 2x_3 & = & 1 \\
   & x_1 &   &     & - & 3x_3 & \leq & 0 \\
   & x_1 & , & x_2 & , & x_3 & \geq & 0.
  \end{array}
  \]
\item
  Write down the dual problem of the following: \[
  \begin{array}{rrcrcrcl}
  \min  & x_1 &  &  & - & 9x_3 \\
  \mbox{s.t.}
   & x_1 & - & 3x_2 &  + & 2x_3 & = & 1 \\
   & x_1 & &  & &  & \leq & 0 \\
   & &  & x_2 & &  &  & \mbox{free} \\
   & &  & &  & x_3 & \geq & 0.
  \end{array}\]
\item
  Determine all values \(c_1,c_2\) such that the linear programming
  problem \[\begin{array}{rl}
  \min & c_1 x_1 + c_2 x_2 \\
  \text{s.t.} & 2x_1 + x_2 \geq 2 \\
  & x_1 + 3x_2 \geq 1.
  \end{array}
  \] has an optimal solution. Justify your answer
\end{enumerate}

\subsection*{Solutions}\label{solutions-6}
\addcontentsline{toc}{section}{Solutions}

\begin{enumerate}
\def\labelenumi{\arabic{enumi}.}
\item
  The dual is \[\begin{array}{rrcrcll}
  \max & 3y_1 \\
  \text{s.t.} 
  & y_1 & +  & 3y_2 & = & 4\\
  & 2y_1 & - & 4y_2 & \leq & -2 \\
  &  y_1 &   &      &\geq & 0.
  \end{array}\]
\item
  The dual is \[\begin{array}{rrcrcll}
  \max  & y_1  &   &   \\
  \mbox{s.t.}
   & y_1 & + &  y_2 & \leq & 0 \\
   & y_1 &   &      & \leq & 3 \\
   &2y_1 & - & 3y_2 & \leq & 1 \\
   & y_1 &   &      &      & \mbox{free} \\
   &     &   &  y_2 &  \leq & 0.
  \end{array}\]
\item
  The dual is \[\begin{array}{rrcll}
  \max  & y_1  \\
  \mbox{s.t.}
   &   y_1 & \geq & 1 \\
   & -3y_1 & = & 0 \\
   & 2y_1 & \leq & -9 \\
   & y_1 &    & \mbox{free}.
  \end{array}\]
\item
  Let (P) denote the given linear programming problem.

  Note that
  \(\begin{bmatrix} x_1 \\ x_2\end{bmatrix} = \begin{bmatrix} 1 \\ 0\end{bmatrix}\)
  is a feasible solution to (P). Therefore, by Theorem \ref{fund-lp}, it
  suffices to find all values \(c_1,c_2\) such that

  \begin{enumerate}
  \def\labelenumii{(\Alph{enumii})}
  \setcounter{enumii}{15}
  \tightlist
  \item
    is not unbounded. This amounts to finding all values \(c_1,c_2\)
    such that the dual problem of (P) has a feasible solution.
  \end{enumerate}

  The dual problem of (P) is \[\begin{array}{rl}
  \max & 2 y_1 + y_2 \\
  & 2y_1 + y_2 = c_1 \\
  & y_1 + 3y_2 = c_2 \\
  & y_1 , y_2 \geq 0. \\
  \end{array}
  \]

  The two equality constraints gives
  \(\begin{bmatrix} y_1 \\ y_2 \end{bmatrix} = \begin{bmatrix} \frac{3}{5}c_1 - \frac{1}{5} c_2 \\ -\frac{1}{5}c_1 + \frac{2}{5} c_2 \end{bmatrix}.\)
  Thus, the dual problem is feasible if and only if \(c_1\) and \(c_2\)
  are real numbers satisfying
  \begin{align*}
  \frac{3}{5}c_1 - \frac{1}{5} c_2 & \geq & 0 \\
  -\frac{1}{5}c_1 + \frac{2}{5} c_2  & \geq & 0,
  \end{align*}

  or more simply, \[\frac{1}{3} c_2 \leq c_1 \leq 2 c_2.\]
\end{enumerate}
