% Copyright 2020 by Robert Hildebrand
%This work is licensed under a
%Creative Commons Attribution-ShareAlike 4.0 International License (CC BY-SA 4.0)
%See http://creativecommons.org/licenses/by-sa/4.0/

%\documentclass[../open-optimization/open-optimization.tex]{subfiles}
%
%\begin{document}

\chapter{Algorithms to Solve Integer Programs}
\todoChapter{ 50\% complete. Goal 80\% completion date: September 20\\
Notes: }
\label{sec:IP-algorithms}

\begin{outcome}
\begin{enumerate}
\item Understand misconceptions in difficulty of integer programs
\item Learn basic concepts of algorithms used in solvers
\item Practice these basic concepts at an elementary level
\item Apply these concepts to understanding output from a solver
\end{enumerate}
\end{outcome}

In this section, we seek to understand some of the fundamental approaches used for solving integer programs.   These tools have been developed the past 70 years.  As  such, advanced solvers today are incredibly complicated and have many possible settings to hope to solve your problem more effiently.  Unfortunately, there is no single approach that is best for all different problems.

\includefigurestatic[GUROBI Performance on a set of problems while varying different  possible settings.  This plot show the wild variablity of performance of different approaches.  Thus, it is very unclear which is the ``best" method. Furthermore, this plot can look quite different depending on the problem set one is working with.   Altough we will not emphasize determining optimal settings in this text, we want to make clear that the techniques used in solvers are quite complicated and are tuned very carefully.  We will study some elementary versions of techniques used in these solvers.][width=0.90\linewidth][h]{gurobi_performance}

Although there are many tricks used to imrove the solve time, there are thee core elements to solving an integer program: \emph{Presolve}, \emph{Primal techniques}, \emph{Cutting Planes}, and \emph{Branch and Bound}.   

\paragraph{Presolve} contains many tricks to elimate variables, reduce the problem size, and make format the problem into something that might be easier to solve.  We will not focus on this aspect of solving integer programs.

\paragraph{Primal techniques} use a variety of approaches to try to find feasible solutions.  These feasible solutions are extemely helpful in conjunction with branch and bound.

\paragraph{Cutting planes} are ways to improve the description by adding additional inequalities.  There are many ways to derive cutting planes.  We will learn just a couple to get an idea of how these work.  

\paragraph{Brach and bound} is a method to decompose the problem into smaller subproblems and also to certify optimality (or at least provide a bound to how close to optimal a solution is) by removing sets of subproblems that can be argued to be suboptimal.   We will look at an elementary branch and bound approach.  \underline{Understanding this technique is key to explaining the ouput of an integer programming solver.}


We will being this chapter with a comparison of solving the linear programming relaxation comparted to solving an integer program.  We will then use this understanding as fundamental to both the techniques of cutting planes and branch and bound.
We will end this section with an example of output from GUROBI and explain how to interpret this information.

\section{LP to solve IP}

Recall that the linear relaxation of an integer program is the linear programming problem after removing the integrality constraints
$$
\begin{array}{rlclr}
\text{ Integer Program:} & & & \text{Linear Relaxation:}\\
\max \ & z_{IP} = c^\top x & \hspace{3cm} & & \max \  z_{LP} = c^\top x \\
& Ax \leq b & & & Ax \leq b\\
& \tred{x \in \Z^n} & & & \tblue{x \in \R^n}
\end{array}
$$

\begin{theorem}{LP Bounds}{}
It always holds that 
\begin{equation}
z^*_{IP} \leq z^*_{LP}.
\end{equation}
Furthermore, if $x^*_{LP}$ is integral (feasible for the integer program), then 
\begin{equation}
x^*_{LP} = x^*_{IP} \ \ \text{ and } z^*_{LP} = z^*_{IP}.
\end{equation}
\end{theorem}

\begin{example}{}{}

%%second column
%\begin{minipage}[t]{0.1\textwidth}
%\includegraphics[scale = 0.3]{LP-equals-IP.png}
%\end{minipage}
%
%Here is some text.\\
% first column
\begin{minipage}[t]{0.5\textwidth}
Consider the problem 
\begin{align*}
\max z = & 3x_1 + 2x_1\\
& 2x_1 + x_2 \leq 6\\
& x_1, x_2 \geq 0; x_1, x_2 \text{ integer}
\end{align*}
\end{minipage}
%
%second column
\begin{minipage}[t]{0.4\textwidth}
%This is the graph
\end{minipage}
\end{example}


%  \includegraphics[width=2cm,height=2cm]{LP-equals-IP.png}\fbox{\LARGE J}
  
  
  \subsection{Rounding LP Solution can be bad!}

  Consider the two variable knapsack problem
  \begin{align}
  \max   3x_1 + 100 x_2\\
              x_1  + 100 x_2 \leq 100\\
              x_i \in \{0,1\} \text{ for } i=1,2.
  \end{align}
  
  Then $x^*_{LP} = (1, 0.99)$ and $z^*_{LP} = 1\cdot 3 + 0.99\cdot 100 = 3 + 99 = 102.$
  
  But $x^*_{IP} = (0,1)$ with $z^*_{IP} = 0\cdot 3 + 1 \cdot 100 = 100$.
  
  Suppose that we rounded the LP solution.  
  
  $x^*_{LP-Rounded-Down} = (1,0)$.  Then $z^*_{LP-Rounded-Down} = 1\cdot 3 = 3$.  Which is a terrible solution!
  
  
  How can we avoid this issue?
  
  
  Cool trick!   Using two different strategies gives you at least a 1/2 approximation to the optimal solution.
  
  
  \subsection{Rounding LP solution can be infeasible!}
  Now only could it produce a poor solution, it is not always clear how to round to a feasible solution.  
  
\subsection{Fractional Knapsack}
The fractional knapsack problem has an exact greedy algorithm.

\section{Branch and Bound}


\subsection{Algorithm}


\begin{algorithm}[H]
\algorithmicrequire{Integer Linear Problem with max objective}\\
\algorithmicensure{Exact Optimal Solution $x^*$}
\caption{Branch and Bound - Maximization}\label{alg:branch-and-bound-max}
\begin{algorithmic}[1]
	\State Set $LB = - \infty$.
 	\State Solve LP relaxation. 
	\begin{algsubstates}
        		\State If $x^*$ is integer, stop!
        		\State Otherwise, choose fractional entry $x_i^*$ and branch onto subproblems:
		(i) $x_i \leq \lfloor x^*_i \rfloor$ and 
		(ii) $x_i \geq \lceil x^*_i \rceil$.    
	   \end{algsubstates}
	\State Solve LP relaxation of any subproblem.
		\begin{algsubstates}
		\State If LP relaxation is infeasible, prune this node as \textbf{"Infeasible"}
        		\State If $z^* < LB$, prune this node as \textbf{"Suboptimal"}
		\State $x^*$ is integer, prune this nodes as \textbf{"Integer"} and update $LB = \max(LB, z^*)$.
		\State Otherwise, choose fractional entry $x_i^*$ and branch onto subproblems:  (i) $x_i \leq \lfloor x^*_i \rfloor$ and (ii) $x_i \geq \lceil x^*_i \rceil$.     Return to step 2 until all subproblems are pruned.
      \end{algsubstates}
      \State Return best integer solution found.
	\end{algorithmic}
\end{algorithm}

%\url{http://www.sce.carleton.ca/faculty/chinneck/po/Chapter12.pdf}
%
%\url{http://www.sce.carleton.ca/faculty/chinneck/po/Chapter13.pdf}
%
%\includegraphics[scale = 0.5]{branch-and-bound-problem}
%
%\includegraphics[scale = 0.5]{branch-and-bound}




Here is an example of branching on general integer variables.

\begin{example}{}{}
Consider the two variable example with
 
 \begin{align*}
 \max & -3x_1 + 4x_2\\
 & 2x_1 + 2 x_2 \leq 13\\
 & -8 x_1 + 10x_2 \leq 41\\
 & 9x_1 + 5x_2 \leq 45\\
 & 0 \leq x_1 \leq 10, \text{ integer }\\
 & 0 \leq x_2 \leq 10, \text{ integer }
 \end{align*}
$x =  [1.33, 5.167]  \textrm{obj} =  16.664$\\
\noindent \includegraphicstatic[scale = 0.3]{branch-and-bound1}
$x =  [1,  4.9]  \textrm{obj} =  16.5998$\\
$x =  [2,  4.5]  \textrm{obj} =  12.0$\\
\includegraphicstatic[scale = 0.3]{branch-and-bound2}

\end{example}

\begin{example}{Example continued}{}
Infeasible Region\\
$x =  [0, 4]  \textrm{obj} =  16.0$\\
$x =  [2,  4.5]  \textrm{obj} =  12.0$\\
\includegraphicstatic[scale = 0.3]{branch-and-bound3}

\end{example}

% Branch and bound tree for maximize knapsack
\tikzset{
  S/.style = {draw, rectangle, rounded corners=0.1cm, minimum size = 8mm,  top color=blue!10, bottom color=blue!10},% top color=white, bottom color=blue!20},
}
     \begin{tikzpicture}[-,thick]
\node[S,sibling distance=5cm, level distance=1.3cm, align=left, label={[blue]left:$t = 1$}, label = {above:\textbf{Root Node}}] 
	{$x^* = (1.33, 5.167) $\\ $z^* = 16.664$\\ $ LB = -\infty$} 
	[sibling distance=7cm, level distance=2.8cm,align=left]
	%
     	child {node[S, label={[blue]left:$t = 2$}] 
     		{$x^* = (1, 4.9) $\\$z^* = 16.5998$\\$ LB = -\infty$}
		[sibling distance=5cm, level distance=2.8cm,align=left]
		child {node[S, label={[blue]left:$t = 4$}, label = {[red]below: \rule{3cm}{0.8pt} \\ \centering Integer}] {$x^* = (0,4)$\\$z^* = 16$\\$ LB = 16$}
		%
     			edge from parent node[left] {$x_2 \leq 4$}}
			%
			child {node[S, label={[blue]right:$t =5$}, label = {[red]below: \rule{3cm}{0.8pt} \\ \centering Infeasible}] {$x^* = \textrm{N/A}$\\$z^* = \textrm{N/A}$\\$ LB = \textrm{N/A}$}[sibling distance=5cm, level distance=2.8cm,align=left]
			%
			edge from parent node[right] {$x_2 \geq 5$}
			}
%
   		edge from parent node[left] {$x_1 \leq 1$}
   }
     child {node[S, label={[blue]right:$t =3$},label = {[red]below: \rule{3cm}{0.8pt} \\ \centering Suboptimal}] 
     		{$x^* = (2, 4.5) $\\$z^* = 12.0$\\$ LB = -\infty$}
		[sibling distance=5cm, level distance=2.8cm,align=left]
		%
   		edge from parent node[right] {$x_1 \geq 2$}
   };

     \end{tikzpicture}


\subsection{Knapsack Problem  and 0/1 branching}

Consider the problem 

\begin{align*}
\max \quad & 16x_1 + 22x_2 + 12x_3 + 8 x_4\\
\st & 5x_1 + 7x_2 + 4x_3 + 3x_4 \leq 14\\
& 0 \leq x_i \leq 1 \ \ i=1,2,3,4\\
& x_i \in \{0,1\} \ \ i=1,2,3,4
\end{align*}

\textbf{Question: What is the optimal solution if we remove the binary constraints?}




\begin{align*}
\max \quad & c_1 x_1 + c_2 x_2 + c_3 x_3 + c_4 x_4\\
\st & a_1 x_1 + a_2 x_2 + a_3 x_3 + a_4 x_4 \leq b\\
& 0 \leq x_i \leq 1 \ \ i=1,2,3,4\\
\end{align*}

\textbf{Question: How do I find the solution to this problem?}


\begin{align*}
\max \quad & c_1 x_1 + c_2 x_2 + c_3 x_3 + c_4 x_4\\
\st & (a_1 - A)x_1 + (a_2-A) x_2 + (a_3-A) x_3 + (a_4-A) x_4 \leq 0\\
& 0 \leq x_i \leq m_i \ \ i=1,2,3,4\\
\end{align*}

\textbf{Question: How do I find the solution to this problem?}



Consider the problem 

\begin{align*}
\max \quad & 16x_1 + 22x_2 + 12x_3 + 8 x_4\\
\st & 5x_1 + 7x_2 + 4x_3 + 3x_4 \leq 14\\
& 0 \leq x_i \leq 1 \ \ i=1,2,3,4\\
& x_i \in \{0,1\} \ \ i=1,2,3,4
\end{align*}

We can solve this problem with branch and bound.


The optimal solution was found at $t=5$ at subproblem 6 to be $x^* = (0,1,1,1)$, $z^* = 42$.



\textbf{Example: Binary Knapsack}
%\begin{example}{Binary Knapsack Example}{}
Solve the following problem with branch and bound.
\begin{align*}
\max\ \ \   z&=11x_1+15x_2+6x_3+2x_4 + x_5\\
\text{Subject to:} \ \ \ 	 &5x_1+7x_2+4x_3+3x_4 + 15x_5\leq15\\
		&x_i  \text{  binary},i=1,\dots,5
\end{align*}

% Branch and bound tree for maximize knapsack
\tikzset{
  S/.style = {draw, rectangle, rounded corners=0.1cm, minimum size = 8mm,  top color=blue!10, bottom color=blue!10},% top color=white, bottom color=blue!20},
}
     \begin{tikzpicture}[-,thick]
\node[S,sibling distance=5cm, level distance=1.3cm, align=left, label={[blue]left:$t = 1$}, label = {above:\textbf{Root Node}}] 
	{$x^* = (1,1,0.75,0,0) $\\ $z^* = 30.5$\\ $ LB = -\infty$} 
	[sibling distance=7cm, level distance=2.8cm,align=left]
	%
     	child {node[S, label={[blue]left:$t = 2$}, label = {[red]below: \rule{3cm}{0.8pt} \\ \centering Integer}] 
     		{$x^* = (1,1,0,1,0) $\\$z^* = 28$\\$ LB = 28$}
		[sibling distance=2.5cm, level distance=2.8cm,align=left]
   		edge from parent node[left] {$x_3 = 0$}
   }
     child {node[S, label={[blue]right:$t =3$}] 
     		{$x^* = (1,0.857,1,0,0) $\\$z^* = 29.857$\\$ LB = 28$}
		[sibling distance=5cm, level distance=2.8cm,align=left]
		%
     		child {node[S, label={[blue]left:$t = 4$}, label = {[red]below: \rule{3cm}{0.8pt} \\ \centering Suboptimal}] {$x^* = (1,0,1,1,0.2)$\\$z^* = 19.2$\\$ LB = 28$}
     			edge from parent node[left] {$x_2 = 0$}}
     		child {node[S, label={[blue]right:$t =5$}] {$x^* = (0.8,1,1,0,0)$\\$z^* = 29.8$\\$ LB = 28$}[sibling distance=5cm, level distance=2.8cm,align=left]
			child {node[S, label={[blue]left:$t =6$}, , label = {[red]below: \rule{3cm}{0.8pt} \\ \centering Suboptimal}] {$x^* = (0,1,1,1,0.67)$\\$z^* = 23.067$\\$ LB = 28$}[sibling distance=5cm, level distance=2.8cm,align=left]
				edge from parent node[left] {$x_1 = 0$}
				}
			child {node[S, label={[blue]right:$t =7$}, label = {[red]below: \rule{3cm}{0.8pt} \\ \centering Infeasible}] {\\ Infeasible \hspace{3cm} \\ }[sibling distance=5cm, level distance=2.8cm,align=left]
				edge from parent node[right] {$x_1 = 1$}
				}
     			edge from parent node[right] {$x_2 = 1$}
			}
   		edge from parent node[right] {$x_3 = 1$}
   };

     \end{tikzpicture}


%\end{example}

\subsection{Traveling Salesman Problem solution via Branching}


\todo[inline]{
Describe solving TSP via a generalized branching method that removes subtours (instead of adding constraints).
}




\section{Cutting Planes}
Cutting planes are inequalities $\pi^\top x \leq \pi_0$ that are valid for the feasible integer solutions that the cut off part of the LP relaxation.  Cutting planes can create a tighter description of the feasible region that allows for the optimal solution to be obtained by simply solving a strengthened linear relaxation. 

The cutting plane procedure, as demonstrated in Figure~\ref{fig:cutting-plane-procudure}.  The procedure is as follows:
\begin{enumerate}
\item Solve the current LP relaxation.
\item If solution is integral, then return that solution.  STOP
\item Add a cutting plane (or many cutting planes) that cut off the LP-optimal solution.
\item Return to Step 1.
\end{enumerate}

\begin{figure}[H]
\includegraphicstatic[scale = 0.4]{figureCutttingPlane1}
\includegraphicstatic[scale = 0.4]{figureCutttingPlane2}
\includegraphicstatic[scale = 0.4]{figureCutttingPlane3}
\includegraphicstatic[scale = 0.4]{figureCutttingPlane4}
\includegraphicstatic[scale = 0.4]{figureCutttingPlane5}
\includegraphicstatic[scale = 0.4]{figureCutttingPlane6}
\includegraphicstatic[scale = 0.4]{figureCutttingPlane7}
\caption{The cutting plane procedure.}
\label{fig:cutting-plane-procudure}
\end{figure}

In practice, this procedure is integrated in some with with branch and bound and also other primal heuristics.

%\begin{table}
%\centering\begin{tabular}{|>{\centering\arraybackslash}m{3cm}|>{\centering\arraybackslash}m{5cm}|} \hline\textbf{Topic} & \textbf{Paragraph} \\\hline \hlineTopic 1 & This is a paragraph. This is a paragraph. This is a paragraph.\\\hline\end{tabular}
%\end{table}

\begin{table}[h]
\centering\begin{tabular}{>{\centering\arraybackslash}m{5cm}>{\centering\arraybackslash}m{10cm}}
 \hline
\textbf{Model} & \textbf{LP Solution} \\\hline \hline

$$
\begin{array}{lrl}
\max \ & x_1 + x_2\\
\text{subject to  } &  -2x_1 + x_2 &\leq 0.5\\
& x_1 + 2x_2 &\leq 10.5\\
& x_1 - x_2 &\leq 0.5\\
& - 2x_1 - x_2 &\leq -2
\end{array} 
$$
&
\includegraphicstatic[scale = 0.5]{cutting-plane-1-picture}\\
$$
\begin{array}{lrl}
\max \ & x_1 + x_2\\
\text{subject to  } &  -2x_1 + x_2 &\leq 0.5\\
& x_1 + 2x_2 &\leq 10.5\\
& x_1 - x_2 &\leq 0.5\\
& - 2x_1 - x_2 &\leq -2\\
& \tred{x_1} &\tred{\leq 3}
\end{array} 
$$
&
\includegraphicstatic[scale = 0.5]{cutting-plane-2-picture}\\
$$
\begin{array}{lrl}
\max \ & x_1 + x_2\\
\text{subject to  } &  -2x_1 + x_2 &\leq 0.5\\
& x_1 + 2x_2 &\leq 10.5\\
& x_1 - x_2 &\leq 0.5\\
& - 2x_1 - x_2 &\leq -2\\
& \tred{x_1} & \tred{\leq 3}\\
& \tred{x_1 + x_2} & \tred{\leq 6}
\end{array} 
$$
&
\includegraphicstatic[scale = 0.5]{cutting-plane-3-picture}\\
\hline
 \end{tabular}
\end{table}


\subsection{Chv\'atal Cuts}


Chv\'atal Cuts are a general technique to produce new inequalities that are valid for feasible integer points.  


\begin{general}{Chv\'atal Cuts}{}
Suppose 
\begin{equation}
a_1 x_1 + \dots + a_n x_n \leq d
\end{equation}
is a valid inequality for the polyhedron $P = \{ x \in \R^n : Ax \leq b, x \geq 0\}$, then 
\begin{equation}
\label{eq:chvatal}
\lfloor a_1\rfloor x_1 + \dots + \lfloor a_n\rfloor  x_n \leq \lfloor d\rfloor
\end{equation}
is valid for the integer points in $P$, that is, it is valid for the set $P \cap \Z^n$.  Equation~\eqref{eq:chvatal} is called a Chv\'atal Cut.
\end{general}


We will illustrate this idea with an example.


\begin{example}{}{}
Recall example~\ref{ex:min-coins}.  The model was\\
\textbf{Model}
\begin{align*}
\min \quad & p + n + d + q & \text{ total number of coins used}\\
\text{ s.t. } \quad & p + 5n + 10d + 25 q = 83 & \text{sums to } 83 \cent\\
& p,d,n,q \in \Z_+ & \text{each is a non-negative integer}
\end{align*}

From the equality constraint we can derive several inequalities.
\begin{enumerate}
\item Divide by 25 and round down both sides:
\[
\frac{p + 5n + 10d + 25 q}{25} = 83/25 \quad \Rightarrow \quad q \leq 3 
\]
\item Divide by 10 and round down both sides:
\[
\frac{p + 5n + 10d + 25 q}{10} = 83/10 \quad \Rightarrow \quad d + 2q \leq 8 
\]
\item Divide by 5 and round down both sides:
\[
\frac{p + 5n + 10d + 25 q}{10} = 83/5 \quad \Rightarrow \quad n + 2d  + 5q \leq 16
\]
\item Multiply by 0.12 and round down both sides:
\[
0.12(p + 5n + 10d + 25 q = 0.12 (83) \quad \Rightarrow \quad d  + 3q \leq 9
\]
\end{enumerate}
These new inequalities are all valid for the integer solutions.  Consider the new model:\\

\textbf{New Model}
\begin{align*}
\min \quad & p + n + d + q & \text{ total number of coins used}\\
\text{ s.t. } \quad & p + 5n + 10d + 25 q = 83 & \text{sums to } 83 \cent\\
& q \leq 3\\
& d + 2q \leq 8 \\
& n + 2d  + 5q \leq 16\\
& d  + 3q \leq 9\\
& p,d,n,q \in \Z_+ & \text{each is a non-negative integer}
\end{align*}

The solution to the LP relaxation is exactly $q = 3, d = 0, n = 1, p = 3$, which is an integral feasible solution, and hence it is an optimal solution.
\end{example}


\subsection{Gomory Cuts}
Gomory cuts are a type of Chv\'atal cut that is derived from the simplex tableau.  Specifically, suppose that 
\begin{equation}
\label{eq:tableau-row}
 x_i + \sum_{i\in N} \tilde a_i x_i = \tilde b_i
\end{equation}
is an equation in the optimal simplex tableau. 

\begin{general}{Gomory Cut}{}
The Gomory cut corresponding to the tableau row ~\eqref{eq:tableau-row} is

\begin{equation}
\label{eq:gomory-cut}
\sum_{i\in N} (\tilde a_i - \lfloor \tilde a_i \rfloor) x_i \geq \tilde b_i - \lfloor \tilde b_i\rfloor
\end{equation}


\end{general}


We will solve the following problem using only Gomory Cuts.
\begin{equation*}
\begin{array}{lrcl}
\min & x_1 - 2x_2\\
\st & -4x_1 + 6x_2  & \leq & 9\\
& x_1 + x_2   & \leq & 4\\
& x \geq 0 & , & x_1,x_2 \in \Z
\end{array}
\end{equation*}

\textbf{Step 1:} The first thing to do is to put this into standard from by appending slack variables.
\begin{equation}
\label{eq:gomory-standard-form}
\begin{array}{lrcl}
\min & x_1 - 2x_2\\
\st & -4x_1 + 6x_2 + s_1 & = & 9\\
& x_1 + x_2 + s_2  & = & 4\\
& x \geq 0 & , & x_1,x_2 \in \Z
\end{array}
\end{equation}

We can apply the simplex method to solve the LP relaxation.\\
\begin{tabular}{cc}
Initial Basis & 
%\begin{center}
\begin{tabular}{|lr|rrrr|}
\hline
 Basis & RHS & $x_1$ & $x_2$ & $s_1$ & $s_2$ \\
 \hline
 $z$     & 0.0 & 1.0   & -2.0  & 0.0   & 0.0   \\
 \hline
 $s_1$ & 9.0 & -4.0  & 6.0   & 1.0   & 0.0   \\
 $s_2$ & 4.0 & 1.0   & 1.0   & 0.0   & 1.0   \\
\hline
\end{tabular}\\
%\end{center}\\
\vdots & \vdots  \\
Optimal Basis & 
%Pivoting to the optimal basis, we have
%\begin{center}
\begin{tabular}{|lr|rrrr|}
\hline
 Basis & RHS  & $x_1$ & $x_2$ & $s_1$ & $s_2$ \\
 \hline
 $z$     & -3.5 & 0.0   & 0.0   & 0.3   & 0.2   \\
 \hline
 $x_1$ & 1.5  & 1.0   & 0.0   & -0.1  & 0.6   \\
 $x_2$ & 2.5  & 0.0   & 1.0   & 0.1   & 0.4   \\
\hline
\end{tabular}
%\end{center}
\end{tabular}

This LP relaxation produces the fractional basic solution $x_{LP} = (1.5, 2.5)$.


\begin{example}{}{}\textbf{(Gomory cut  removes LP solution)}{}
 We now identify an integer variable $x_i$ that has a fractional basic solution.  Since both variables have fractional values, we can choose either row to make a cut.  Let's focus on the row corresponding to $x_1$.

The row from the tableau expresses the equation 
\begin{equation}
x_1 - 0.1 s_1 + -0.6 s_2 = 1.5.
\end{equation}

Applying the Gomory Cut~\eqref{eq:gomory-cut}, we have the inequality 
\begin{equation}
\label{eq:gomory-cut-ex1}
0.9 s_1 + 0.4 s_2 \geq 0.5.
\end{equation}

The current LP solution is $(x_{LP}, s_{LP}) = (1.5, 2.5,0,0)$.  Trivially, since $s_1, s_2 = 0$, the inequality is violated.
\end{example}

\begin{example}{\textbf{(Gomory Cut in Original Space)}}{}

The Gomory Cut~\eqref{eq:gomory-cut-ex1} can be rewritten in the original variables using the equations from~\eqref{eq:gomory-standard-form}.  That is, we can use the equations
\begin{equation}
\label{eq:gomory-standard-form-equations}
\begin{array}{lrcl}
 & s_1 & = & 9 + 4x_1 - 6x_2\\
&s_2  & = & 4 - x_1 - x_2,
\end{array}
\end{equation}
which transforms the Gomory cut into the original variables to create the inequality
\begin{equation*}
\label{eq:gomory-cut-ex1-original}
0.9 (9 + 4x_1 - 6x_2) + 0.4(4 - x_1 - x_2) \geq 0.5.
\end{equation*}
or equivalently
\begin{equation}
\label{eq:gomory-cut-ex1-original}
- 3.2 x_1 + 5.8 x_2 \leq 9.2.
\end{equation}
%\textbf{Check!} We can now check if this inequality cuts off the current LP optimal solution.
%To see this, let's plug in $(x_{LP}, s_{LP}) = (1.5, 2.5,0,0)$.
%
%\begin{equation}
%\label{eq:gomory-cut-ex1-original}
%- 3.2 x_1 + 5.8 x_2 = - 3.2 (1.5) + 5.8 (2.5) = 9.7 > 9.2
%\end{equation}
%Thus, this LP solution is violated by the new Gomory Cut.

As you can see, this inequality does cut off the current LP relaxation.
%\includegraphics[scale = 0.5]{this-graphic.}

\end{example}


\begin{example}{\textbf{(Gomory  cuts plus new tableau)}}{}
Now we add the slack variable $s_3 \geq 0$ to make the equation
\begin{equation}
0.9 s_1 + 0.4 s_2 - s_3 = 0.5.
\end{equation}


\end{example}


Next, we need to solve the linear programming relaxation (where we assume the variables are continuous).

\subsection{Cover Inequalities}
Consider the binary knapsack problem
\begin{align*}
\max \ \ & x_1 + 2x_2 + x_3 + 7 x_4\\
\text{s.t.} \ \ & 100 x_1 + 70x_2 + 50 x_3 + 60 x_4 \leq 150\\
& x_i  \text{ binary  for } i =1, \dots, 4
\end{align*}

A \emph{cover} $S$ is any subset of the variables whose sum of weights exceed the capacity of the right hand side of the inequality.\\
For example, $S = \{1,2,3,4\}$ is a cover since $100 + 70 + 50 + 60 > 150$.\\
Since not all variables in the cover $S$ can be in the knapsack simultaneously, we can enforce the \emph{cover inequality}
\begin{equation}
\sum_{i \in S} x_i \leq |S| - 1 \ \ \ \Rightarrow \ \ \ \ x_1 + x_2 + x_3 + x_4 \leq 4 - 1 = 3.
\end{equation}
Note, however, that there are other covers that use fewer variables.\\

A \emph{minimal cover} is a subset of variables such that no other subset of those variables is also a cover.  For example, consider the cover $S' = \{1,2\}$.  This is a cover since $100 + 70 > 150$.  Since $S'$ is a subset of $S$, the cover $S$ is not a minimal cover.  In fact, $S'$ is a minimal cover since there are no smaller subsets of the set $S'$ that also produce a cover.
In this case, we call the corresponding inequality a \emph{minimal cover inequality}.  That is, the inequality
\begin{equation}
x_1 + x_2 \leq 2 - 1 = 1
\end{equation}
is a minimal cover inequality for this problem.   The minimal cover inequalities are the "strongest" of all cover inequalities.

\emph{Find the two other minimal covers (one of size 2 and one of size 3) and write their corresponding minimal cover inequalities.}

\begin{solution}{}{}{}
The other minimal covers are 
\begin{equation}
S = \{1,4\} \ \ \Rightarrow \ \ \ x_1 + x_4 \leq 1
\end{equation}
and
\begin{equation}
S = \{2,3,4\} \ \ \Rightarrow \ \ \ x_2 + x_3 + x_4 \leq 2
\end{equation}
\end{solution}


\section{Interpreting Output Information and Progress}
\todoSection{Write this section.  Include screenshot of a solver log}


\includefigurestatic[This shows the progress of the solver over time.][width=0.90\linewidth][h]{solve_progress1}

\includefigurestatic[This shows the progress of the solver over time.][width=0.90\linewidth][h]{solve_progress2}

\section{Branching Rules}
These are advanced techniques that are not necessary at this point.

There is a few clever ideas out there on how to choose which variables to branch on.  We will not go into this here.  But for the interested reader, look into 
\begin{itemize}
\item Strong Branching
\item Pseudo-cost Branching
\end{itemize}

\section{Lagrangian Relaxation for Branch and Bound}
This is an advanced technique that is not necessary to learn at this point.

At each note in the branch and bound tree, we want to bound the objective value.  One way to get a a good bound can be using the Lagrangian. 

See~\cite{Fisher2004}  \href{https://my.eng.utah.edu/~kalla/phy_des/lagrange-relax-tutorial-fisher.pdf}{(link)} for a description of this.




\section{Benders Decomposition}
This is an advanced technique that is not necessary to learn at this point.

\section{Literature and Resources}


\begin{resource}
LP Rounding
\begin{itemize}
\item  \href{https://youtu.be/Az7HUuq4SOI?t=189}{Video! - Michel Belaire (EPFL) looking at rounding the LP solution to an IP solution}
\end{itemize}

Fractional Knapsack problem
\begin{itemize}
\item \href{https://youtu.be/m1p-eWxrt6g}{Video solving the Fractional Knapsack Problem}
\item \href{https://www.geeksforgeeks.org/fractional-knapsack-problem/}{Blog solving the Fractional Knapsack Problem}
\end{itemize}

Branch and Bound
\begin{itemize}
\item \href{https://www.youtube.com/watch?v=SdXPNaID-T8}{Video! -  Michel Belaire (EPFL) Teaching Branch and Bound Theory}
\item \href{https://www.youtube.com/watch?v=nKXZYQUtvAY}{Video! -  Michel Belaire (EPFL) Teaching Branch and Bound with Example}
\item See \href{http://web.tecnico.ulisboa.pt/mcasquilho/compute/_linpro/TaylorB_module_c.pdf}{Module by Miguel Casquilho} for some nice notes on branch and bound.
\end{itemize}

Gomory Cuts
\begin{itemize}
\item \href{https://www.youtube.com/watch?v=1i0rKtH_YPs&list=PLNMgVqt8MREx6Nex1Q9003vrZem-JXNvX&index=28&ab_channel=EducationalDocumentaries}{Pascal Van Hyndryk (Georgia Tech) Teaching Gomory Cuts}
\item \href{https://www.youtube.com/watch?v=VdXHGNDnjjo}{Michel Bierlaire (EPFL) Teaching Gomory Cuts}
\end{itemize}

Benders Decomposition
\begin{itemize}
\item \href{https://www.juliaopt.org/notebooks/Shuvomoy%20-%20Benders%20decomposition.html}{Benders Decomposition - Julia Opt}
\item \href{https://www.youtube.com/watch?v=8vUNXHwVnC8}{Youtube!  SCIP lecture}
\end{itemize}
\end{resource}


%\end{document}
